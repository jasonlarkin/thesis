\documentclass[letterpaper,12pt]{article}
\usepackage[dvips]{epsfig,geometry}
%\usepackage{multicol}
%\usepackage{fancyhdr}
%\pagestyle{fancy}
\usepackage{nicefrac}
\usepackage{indentfirst}
\usepackage{amssymb}
\usepackage{color}
\usepackage{enumerate}
\usepackage{comment}
\usepackage{overcite}
\usepackage{citesort}
\usepackage{geometry}
\usepackage{amsmath}
\usepackage[]{caption}
%\usepackage{aip}
\usepackage{bm}
\usepackage{wrapfig}
\usepackage{subfigure}
\geometry{letterpaper,textwidth=6.5in,textheight=9.0in,left=1.in,right=1in,top=1in}
\setlength{\parindent}{0.25in}
\renewcommand{\baselinestretch}{1.25}
\baselineskip = 13pt

\columnsep = 0.125in

%Definition of new commands
\newcommand{\f}[2]{\ensuremath{\frac{\displaystyle{#1}}{\displaystyle{#2}}}}
\newcommand{\lr}[1]{\langle{#1}\rangle}
\newcommand{\colv}[2] {\left(\begin{array}{c} #1 \\ #2 \end{array}\right)}
\renewcommand{\thefootnote}{\fnsymbol{footnote}}



\begin{document}

\begin{center}



\centering

\vspace{1.1in}

\LARGE Thermal Transport in Disordered Materials\Large

\vspace{1in} A thesis proposal by\\Jason M. Larkin\\
\vspace{1.1in}


\parbox[h]{4in}{\center{April 25, 2008\\224 Scaife Hall\\Department of
Mechanical Engineering\\Carnegie Mellon University}}

\thispagestyle{empty}
\end{center}
\vspace{2.in}
\parbox[b]{6.5in}{\noindent \underline{Thesis Committee}\\
\noindent Associate Professor Alan McGaughey (Committee Chair), Mechanical Engineering\\
Professor Jonathan A. Malen, Mechanical Engineering \\
Professor Michael Widom, Physics \\
Professor Craig Maloney, Civil and Environmental Enginering\\}

\clearpage

\tableofcontents

\clearpage

\section{\label{S-Intro}Introduction}

\subsection{\label{S-Intro-Motivation}Motivation}

Superlattices are periodic nanostructures containing alternating
material layers with thicknesses as small as a few nanometers (see
Fig.
\ref{F-SL_picture}).\cite{capinski1999,chakraborty2003,caylor2005,ezzahri2005,borca-tasciuc2000,borca-tasciuc2002,huxtablethesis,venkata2000}

 \textit{It is
likely that the thermoelectric performance potential of
superlattices has yet to be fully realized.}

%\begin{figure}[h]
%\centerline{\epsfig{figure=SL_picture}}
%\caption{\label{F-SL_picture} \small{Image of
%Si/Si$_{0.78}$Ge$_{0.22}$ superlattice obtained by transmission
%electron microscopy.\cite{aubertine2005}}}
%\end{figure}

\subsection{\label{S-Intro-Review}Literature review}

\subsubsection*{Superlattice thermal conductivity}

Superlattices can have high values of the thermoelectric
figure-of-merit because of their low cross-plane thermal
conductivities, which in some cases, have been observed to be less
than the that of an alloy of similar
composition.\cite{capinski1999,chakraborty2003,borca-tasciuc2000,borca-tasciuc2002,venkata2000,caylor2005,lee1997}
For some of these cases, the reduction below the alloy thermal
conductivity is due to defects and dislocations that result from the
strain associated with the lattice mismatch.\cite{kim2006,kim2007}
Reductions below the alloy thermal conductivity have also been
observed, however, for superlattices without significant defects or
dislocations.\cite{kim2007,capinski1999,venkata2000,borca-tasciuc2002}
The effect of the superlattice period length on the experimentally
observed thermal conductivity is conflicting. Some studies have
found that the thermal conductivity initially decreases with
increasing period length until a minimum is reached, beyond which
the thermal conductivity
increases.\cite{chakraborty2003,venkata2000,caylor2005} Others have
observed the thermal conductivity to increase monotonically with
increasing period length.\cite{capinski1999,huxtablethesis}

Lattice dynamics calculations,
\cite{simkin2000,yang2003,broido2004,yang2001,ren2006,tamura1999,kiselev2000,hyldgaard1997,bies2000}
and molecular dynamics (MD) simulations
\cite{abramson2002,chen2005,daly2002,imamura2003,volz2000,mcgaughey2006,chiritescu2007}
have been applied to investigate the experimental trends.
Traditional lattice dynamics calculations require the assumption of
coherent phonon transport. This assumption is only valid when the
superlattice period length is not larger than the phonon mean free
path. Lattice dynamics calculations performed under this assumption
have been used to explain the trend of decreasing thermal
conductivity with increasing period length that is sometimes
observed for small-period
superlattices.\cite{kiselev2000,tamura1999,yang2001,bies2000} This
trend is due to the formation of minibands (i.e., frequency gaps) in
the phonon dispersion and reductions in the average phonon group
velocity as the superlattice period length increases. Most of the
traditional lattice dynamics-based approaches make the additional
assumption that all phonon modes have the same relaxation time
[i.e., the constant relaxation time approximation
(CRTA)].\cite{yang2001,ren2006,tamura1999,kiselev2000,hyldgaard1997,bies2000}
Broido and Reinecke \cite{broido2004} used an anharmonic lattice
dynamics model to obtain the mode-specific relaxation times for
Si/Ge superlattices, finding a thermal conductivity trend similar to
that obtained under the CRTA. They note, however, that the agreement
between the trends is fortuitous and due to a cancelation of errors
in the CRTA approach. When the superlattice period length exceeds
the phonon mean free path, the phonon transport is incoherent (i.e.,
phonons scatter at the superlattice interfaces and the phonon
properties are distinct within each layer). In this regime, the
mechanism of thermal conductivity reduction is phonon scattering at
the interfaces. Therefore, the thermal conductivity increases with
increasing period length due to decreasing interface density. Simkin
and Mahan \cite{simkin2000} incorporated the effects of incoherent
phonon transport in lattice dynamics calculations by giving the
phonons a finite mean free path. This task was accomplished through
the addition of an imaginary component to the wavevector. The
resulting model predicted a minimum in the thermal conductivity,
corresponding to the period length where phonon transport
transitions between the coherent and incoherent regimes. Yang and
Chen \cite{yang2003} later expanded this model by adding a second
imaginary component to the wavevector to account for diffuse phonon
scattering at the interfaces. Qualitative agreement with the
experimental trends for GaAs/AlAs superlattices was obtained.

Chen \textit{et al.} \cite{chen2005} applied MD to model
Lennard-Jones (LJ) superlattices in order to examine the conditions
required to produce a minimum thermal conductivity. It was found
that a minimum exists when there is no lattice mismatch, but when
the species were given a lattice mismatch of 4\%, the thermal
conductivity increased monotonically with increasing period length.
Daly \textit{et al.} \cite{daly2002} and Imamura \textit{et al.}
\cite{imamura2003} predicted the effect of interface roughness on
the minimum thermal conductivity for model GaAs/AlAs superlattices.
Both groups found that the addition of rough interfaces decreased
the thermal conductivity and removed the minimum that they observed
for superlattices with perfect interfaces. Apart from our recent
work,\cite{landry2008a} all of the experimental and theoretical
effort to date has focused on simple superlattice designs with two
layers in the unit cell (e.g., the superlattice shown in Fig.
\ref{F-SL_picture}).

\subsubsection*{Thermal boundary resistance}

In order to design superlattices with low thermal conductivity, an
understanding of the nature of thermal transport across the internal
interfaces is required. While experimental methods exist to measure
the thermal boundary resistance of isolated interfaces (i.e., where
the separation between the interfaces is much greater than the bulk
phonon mean free path) where at least one of the species is a metal,
there is no method to measure the thermal boundary resistance of
semiconductor/semiconductor
interfaces.\cite{costescu2003,stevens2005}

To date, little theoretical work has been performed to predict the
thermal boundary resistance for closely-spaced interfaces. For
isolated interfaces, the two most common theoretical models for the
thermal boundary resistance are the acoustic mismatch and diffuse
mismatch models.\cite{swartz1989}. Both models treat the interface
as a two dimensional object and assume that the phonon properties of
the bulk materials exists right up to the interface. In the acoustic
mismatch model, the mismatch between the interface species is
defined by the ratio of the material's acoustic impedances, where
the acoustic impedance is defined as the product of the density and
sound speed. The phonons are assumed to be continuum elastic waves
in the acoustic mismatch model. In the diffuse mismatch model, the
degree of mismatch is defined in terms of the bulk phonon density of
states. While both models are in reasonable agreement with
experiment at temperatures less than $\sim$30 K, they do not predict
the correct magnitude or temperature dependence of the thermal
boundary resistance at higher temperatures.\cite{swartz1989}

Recent theoretical work has focused on improving the high
temperature-accuracy of the acoustic mismatch and diffuse mismatch
models by incorporating the effects of exact phonon
dispersion,\cite{reddy2005} species mixing,\cite{beechem2007} and
phonon-phonon scattering\cite{prasher2001} at the interface. While
these models have been shown to be in better agreement with
experiment, they typically require fitting parameters and make many
of the same assumptions that are required in the original acoustic
mismatch and diffuse mismatch models.

Molecular dynamics simulations have also been used to predict
thermal transport across isolated interfaces. Stevens \textit{et
al.}\cite{stevens2007} applied MD (and the direct method) to predict
the thermal boundary resistance of isolated interfaces described by
the Lennard-Jones potential. It was found that species mixing at the
interface decreased the thermal boundary resistance by nearly a
factor of two for highly mismatched interfaces. The results also
indicated that inelastic phonon scattering is important at high
temperature, which may explain the failure of the diffuse mismatch
model to predict the correct temperature dependence of the thermal
boundary resistance (the diffuse mismatch model assumes elastic
scattering events\cite{swartz1989}). Molecular dynamics simulations
have also been used to model thermal transport across
silicon\cite{schelling2004} and diamond\cite{angadi2006} grain
boundaries. Schelling \textit{et al.}\cite{schelling2003} applied MD
to model the interactions of single phonon wave packets incident on
an interface at zero temperature. For long wavelength phonons, the
fraction of the incident phonon energy transmitted across the
interface was found to be in agreement with the predictions of the
acoustic mismatch model. For short wavelength phonons, however, the
predictions of the acoustic mismatch model were not in agreement
with the simulation results. This discrepancy is due to the
invalidity of the continuum elastic waves when the phonon wavelength
is on the order of the interatomic distance. In summary, while
thermal transport across isolated interfaces has been examined
theoretically, the current models are not in agreement with
experiment. Furthermore, little effort has been made to examine the
effect of closely-spaced interfaces on the thermal boundary
resistance.

\clearpage

\subsection{\label{S-Intro-Objectives}Objectives and overview}

The objectives of the proposed research are to

\begin{itemize}
\item Quantitatively describe thermal transport across isolated and
closely-spaced Si/Si$_{1-x}$Ge$_x$ interfaces,

\item Identify metrics for low cross-plane thermal conductivity in
superlattices, and

\item Design Si/Si$_{1-x}$Ge$_x$ superlattices for low cross-plane
thermal conductivity, desirable in thermoelectric energy conversion.

\end{itemize}

Silicon- and germanium-based superlattices and interfaces are the
focus of this work because (i) Si$_{1-x}$Ge$_x$ alloys are among the
best commercially available thermoelectric
materials,\cite{bottner2006,chen2003} (ii) the thermal conductivity
of Si/Ge superlattices has been observed to be below an alloy of
similar composition,\cite{lee1997,chakraborty2003,borca-tasciuc2000}
and (iii) suitable interatomic potential functions exist to model
silicon and germanium.

Molecular dynamics simulations will be the primary tool used to
achieve the research objectives. Details related to the MD
simulations and the thermal conductivity and thermal boundary
resistance prediction methodologies are presented in Section
\ref{S-MD}. The prediction methodology is then validated by
comparing the MD predicted and experimentally measured thermal
conductivities for a variety of silicon and germanium-based samples.
This comparison is discussed in Section \ref{S-validation}. The
proposed research is then discussed in Sections
\ref{S-ProposedWork-Interfaces} and \ref{S-ProposedWorkSLs}.

In several places in this proposal (e.g., in the comparison between
the Green-Kubo and direct thermal conductivity predictions methods,
and in motivating the design strategies for reducing the
superlattice thermal conductivity), the results of my preliminary
study\cite{landry2008a} on model Lennard-Jones superlattices are
discussed. In this preliminary study, the atomic interactions were
described using the Lennard-Jones interatomic potential. The simple
form of the this potential allowed for fast simulations, and for the
elucidation of phenomena and development of analysis techniques that
would not be possible in more complicated systems. The model
superlattices were comprised of two species, $A$ and $B$, that
differed only in their mass (i.e., there is no lattice mismatch).
Mass ratios, $R_m$, of two and five were considered, spanning a
typical range for superlattices used in application (e.g., the mass
ratio between germanium and silicon is 2.6).




\clearpage

\section{\label{S-MD}Molecular dynamics simulations}

\subsection{\label{S-MD-intro}Introduction}

In an MD simulation, the Newtonian equations of motion are used to
predict the time history of the positions and velocities of a set of
atoms. Such simulations are an ideal tool for analyzing thermal
transport in superlattices and across interfaces because, unlike
lattice dynamics-based approaches, no assumptions about the nature
of phonon transport (e.g., the CRTA) are required. The only required
input is a suitable interatomic potential to determine the total
system potential energy, $\Phi$, and net force on each atom
(proportional to the spatial derivative of $\Phi$). We use the
Stillinger-Weber interatomic potential because it has been used
extensively to model thermal transport in bulk and composite
materials,\cite{zhao2005,schelling2002,schelling2003,volz2000,bodapati2006,becker2006}
and has been parameterized for both silicon\cite{stillinger1985} and
germanium.\cite{ding1986} In our simulations, the Newtonian
equations of motion are integrated using the velocity Verlet
algorithm with a time step of 0.55 fs.

\subsection{\label{S-MD-SW}Stillinger-Weber interatomic potential}

In the Stillinger-Weber potential, the total system potential energy
is the sum of two- and three-body terms, and is given by
\begin{equation}
\Phi = \sum_i \sum_{j>i} v_2(i,j) + \sum_i \sum_j \sum_{k>j}
v_3(i,j,k), \label{E-SW}
\end{equation}
where $v_2$ and $v_3$ are functions of the positions and species of
atoms $i$, $j$, and $k$ \cite{stillinger1985}. The first term is a
summation over all atom pairs, and the second term is a summation
over all triplets with atom $i$ at the vertex. The two-body term is
\begin{equation}
v_2(i,j) = \left\{ \begin{array}{ll} \epsilon_{ij} A(B y_{ij}^{-p}
- y_{ij}^{-q})\exp[(y_{ij} - c)^{-1}], & \mbox{if $y_{ij} < c$} \\
0 & \mbox{otherwise}, \end{array} \right. \label{E-SWv2}
\end{equation}
where $A$, $B$, $p$, and $q$ are constants and $c$ is the
nondimensional cutoff radius. The variable $y_{ij}$ is a
dimensionless pair separation defined as $r_{ij}/\sigma_{ij}$, where
$r_{ij} = |\mathbf{r}_i - \mathbf{r}_j|$, and $\mathbf{r}_i$ and
$\mathbf{r}_j$ are the positions of atoms $i$ and $j$. The
parameters $\epsilon_{ij}$ and $\sigma_{ij}$ are the energy and
length scales for the atomic pair ($i$,$j$). The three-body term is
\begin{equation}
v_3(i,j,k) = (\epsilon_{ij}\epsilon_{ik})^{1/2} (\lambda_j
\lambda_i^2 \lambda_k)^{1/4} \exp\left[\frac{\gamma}{y_{ij} - c} +
\frac{\gamma}{y_{ik} - c} \right] \left( \cos \theta_{jik} +
\frac{1}{3} \right)^2, \label{E-SWv3}
\end{equation}
for both $y_{ij} < c$ and $y_{ik} < c$, and zero otherwise. In Eq.
(\ref{E-SWv3}), $\theta_{jik}$ is the angle between atoms $i$, $j$,
and $k$ with atom $i$ at the vertex, and $\lambda$ and $\gamma$ are
constants. The energy and length scales for silicon and germanium
were determined by the potential developers by fitting the potential
to the crystal cohesive energy and
density.\cite{stillinger1985,ding1986} The constants $A$, $B$, $p$,
$q$, $c$, $\lambda$, and $\gamma$ were parameterized for silicon by
fitting the simulation predictions of the liquid structure (i.e.,
the radial distribution function) and melting temperature to
experimental data while ensuring that the crystal structure with the
lowest energy is the diamond lattice.\cite{stillinger1985} For
germanium, the values of $A$, $B$, $p$, $q$, $c$, and $\gamma$ were
chosen to be the same as the values for silicon.\cite{ding1986} The
value of $\lambda$ was parameterized by fitting the potential to the
zero-temperature elastic constants.\cite{ding1986} We use the mixing
rules described by Laradji \textit{et al.} to model the Si-Ge
interactions.\cite{laradji1995}

\subsection{\label{S-MD-k}Methods to predict thermal conductivity and thermal boundary resistance}

\subsubsection*{\label{S-MD-overview}Overview}

The Green-Kubo and direct methods are the two most common methods
for predicting thermal conductivity using molecular dynamics
simulations. While the thermal boundary resistance may also be
predicted using a Green-Kubo approach,\cite{mcgaughey2006b} the
direct method is more straight forward for these predictions. In
this section, the Green-Kubo method for predicting thermal
conductivity and the direct method for predicting thermal
conductivity and thermal boundary resistance are discussed. The
results of my recent comparison\cite{landry2008a} between the
Green-Kubo and direct method predictions of the thermal
conductivities of model Lennard-Jones superlattices are then
presented. To my knowledge, that work represented the first
quantitative comparison between the thermal conductivity prediction
methods for nanostructures such as superlattices.

\subsubsection*{\label{S-MD-GK}Green-Kubo method for predicting thermal conductivity}

The Green-Kubo method is an equilibrium molecular dynamics approach
that relates the equilibrium fluctuations of the heat current
vector, \textbf{S}, to the thermal conductivity, $k$, via the
fluctuation-dissipation theorem. The superlattice thermal
conductivity in the $l$-th direction (either the cross-plane or
in-plane direction) is given by \cite{mcquarrie}
\begin{equation}
k_l=\frac{1}{k_{\mathrm{B}}VT^2}\int_0^{\infty}\langle S_l(t)S_l(0)
\rangle dt, \label{E-GK}
\end{equation}
where $t$ is time, $V$ and $T$ are the system volume and
temperature, and $S_l$ and $\langle S_l(t)S_l(0) \rangle$ are the
$l$-th components of the heat current vector and the heat current
autocorrelation function (HCACF).

There are multiple ways to define the heat current
vector.\cite{mcgaughey2006book,ladd1986,julithesis} The most
commonly used definition is
\begin{equation}
\mathbf{S}_1 = \f{d}{dt} \sum_i \mathbf{r}_i E_i, \label{E-Sreal}
\end{equation}
where $E_i$ is the energy of atom $i$, and the summation is over all
of the atoms in the system. In a solid, where there is no net atomic
motion, the heat flux can also be written using the equilibrium
positions ($\mathbf{r}_{i,o}$) as
\begin{equation}
\mathbf{S}_2 = \f{d}{dt} \sum_i \mathbf{r}_{i,o} E_i. \label{E-Seq}
\end{equation}
The thermal conductivity predictions obtained using both definitions
of the heat current vector were compared in my previous
work.\cite{landry2008a} While both definitions result in the same
prediction for the thermal conductivity, the $\mathbf{S_2}$
definition was found to be preferable for solid-phase simulations.
This definition is preferred because strong oscillations that are
present in the HCACF when using the $\mathbf{S_1}$ definition are
avoided. These oscillations were found to complicate the
specification of the thermal conductivity in simulations of several
different material
systems.\cite{che2000,landry2008a,mcgaughey2006,mcgaughey2004b} An
additional benefit is that the heat current vector is less
computationally expensive with the $\mathbf{S_2}$ definition than
the $\mathbf{S_1}$ definition [although not immediately obvious from
Eqs. (\ref{E-Sreal}) and (\ref{E-Seq}), the $\mathbf{S_1}$
definition requires the calculation of the potential energy of each
atom while the $\mathbf{S_2}$ definition does
not\cite{landry2008a}]. All of the Green-Kubo results presented here
were obtained using the $\mathbf{S_2}$ definition of the heat
current vector.

%\begin{figure}[tb]
%\centerline{\epsfig{figure=5x5_GK-2}} \caption{\label{F-5x5_GK}
%\small{In-plane and cross-plane HCACFs for a model $R_m = 2$,
%$5\times5$ Lennard-Jones superlattice (shown on right). The HCACFs
%have been normalized by their initial values. The integrals of the
%HCACFs (the thermal conductivity) and their converged values (dashed
%lines) are shown in the figure inset.}}
%\end{figure}

Two challenges are encountered when applying the Green-Kubo method
to predict the thermal conductivity. The first challenge is properly
addressing the effect of the finite simulation cell-size on the
predicted thermal conductivity. The thermal conductivity may depend
on the size of the simulation cell if there are not enough phonon
modes to accurately reproduce the phonon scattering in the
associated bulk material. This size dependence is removed by
increasing the simulation cell-size until the thermal conductivity
reaches a size-independent value.

The second challenge is accurately specifying the converged value of
the HCACF integral, which is proportional to the thermal
conductivity through Eq. (\ref{E-GK}). The HCACF and its integral
are shown in Fig$.$ \ref{F-5x5_GK} for a model $R_m = 2$, $5\times5$
Lennard-Jones superlattice as an example. Note that specific
superlattices are referred to in the format $a \times b$, where $a$
and $b$ are the number of monolayers of the first and second
materials. Due to noise in the HCACF that may still exist at long
correlation times (even after averaging the results of multiple
simulations), it can be difficult to specify the region where the
HCACF integral has converged. For the structure shown in Fig$.$
\ref{F-5x5_GK}, the in-plane and cross-plane thermal conductivities
are specified by averaging the HCACF integrals between correlation
times of $\sim$50 ps to $\sim$200 ps.

\subsubsection*{\label{S-MD-DM}Direct methods for predicting thermal conductivity and thermal boundary resistance}

The direct method is a nonequilibrium, steady-state approach for
predicting thermal conductivity or thermal boundary
resistance.\cite{schelling2002,mcgaughey2006book} A schematic of the
direct method simulation cell is shown in Fig$.$ \ref{F-DMsimcell}.
The system consists of a sample bordered by hot and cold reservoirs
and fixed boundaries in the $z$-direction. The fixed boundary
regions each contain four layers of fixed atoms in order to prevent
the sublimation of the reservoir atoms. Periodic boundary conditions
are imposed in the $x$- and $y$-directions.

In the direct method, a known heat flux, $q$, is applied across the
sample, causing a temperature profile to develop. The heat flux is
generated by adding a constant amount of kinetic energy to the hot
reservoir and removing the same amount of kinetic energy from the
cold reservoir at every time step using the method described by
Ikeshoji and Hafskjold.\cite{ikeshoji1994} Example temperature
profiles are shown for samples of bulk germanium and an isolated
Si/Ge interface in Figs. \ref{F-DMsimcell}(b) and
\ref{F-DMsimcell}(c). For predictions of the thermal conductivity,
the temperature gradient within the sample region is specified and
the thermal conductivity is determined using the Fourier law,
\begin{equation}
k = \frac{-q}{\partial T/\partial z}. \label{E-kFourier}
\end{equation}
For predictions of the thermal boundary resistance, the temperature
drop at the interface is specified and the thermal boundary
resistance is determined by
\begin{equation}
R = \frac{\Delta T}{q}. \label{E-DM-R}
\end{equation}

%\begin{figure}[tb]
%\centerline{\epsfig{figure=direct}} \caption{\label{F-DMsimcell}
%\small{(a) Schematic of the simulation cell used in the direct
%method. Example temperature profiles for (b) bulk germanium, and (c)
%an isolated Si/Ge interface.}}
%\end{figure}

As with the Green-Kubo method, a challenge associated with the
direct method is obtaining thermal conductivity predictions that are
independent of the simulation cell-size. The size of the simulation
cell may influence the thermal conductivity when it is not
significantly greater than the bulk phonon mean free path. For
example, when the sample length, $L_S$, is on the order of the bulk
phonon mean free path, the amount of phonon scattering at the
boundaries between the reservoirs and the sample is comparable to
that occurring within the sample itself. Furthermore, phonons can
potentially travel from the hot reservoir to the cold reservoir
without scattering (i.e., ballistic transport). Both of these
effects lead to phonon dynamics not representative of the bulk
sample and a dependence between the thermal conductivity and the
sample length.

For the predictions of the thermal conductivity, we apply the
extrapolation procedure described by Schelling \textit{et
al.}\cite{schelling2002} This procedure is based on the prediction
(obtained from Matthiessen's rule and the kinetic theory expression
for thermal conductivity) that the inverse of the thermal
conductivity decreases linearly with the inverse of the sample
length. Therefore, the thermal conductivity corresponding to a
sample of infinite length can be determined by extrapolating to the
$1/L_S \rightarrow 0$ limit. This extrapolation procedure is
illustrated in Fig$.$ \ref{F-kextrapolation} for bulk germanium, the
Si$_{0.5}$Ge$_{0.5}$ alloy, and the $24\times24$ Si/Ge and
$32\times16$ Si/Si$_{0.7}$Ge$_{0.3}$ superlattices. In all cases,
the trend of $1/k$ versus $1/L_S$ is linear, verifying the use of
the extrapolation procedure to remove the finite sample size effect.
For the predictions of the thermal boundary resistance, the finite
sample length-effect is removed by increasing the sample length
until size-independent predictions are obtained. We find that the
thermal boundary resistance of the Si/Si$_{1-x}$Ge$_x$ interfaces is
not dependent on the sample length when the sample length is greater
than or equal to 800 monolayers ($\sim$110 nm).

%\begin{figure}[tb]
%\centerline{\epsfig{figure=kextrapolation}}
%\caption{\label{F-kextrapolation} \small{Inverse of the predicted
%thermal conductivity versus the inverse of the total sample length.
%Finite-size effects are removed by extrapolating to an infinite
%system size ($1/L_{S} \rightarrow 0$).}}
%\end{figure}

\subsubsection*{\label{S-MD-prediction}Quantitative comparison between the Green-Kubo and direct method predictions of the thermal conductivity}

%\begin{figure}[tb]
%\centerline{\epsfig{figure=DM_GK_Comp}}
%\caption{\label{F-DM_GK_Comp} \small{Comparison between the
%cross-plane thermal conductivity predictions obtained by the
%Green-Kubo and direct methods for $R_m = 2$ and $R_m = 5$, $L_A =
%L_B$ Lennard-Jones superlattices. The cross-plane thermal
%conductivity is plotted as a function of total period length ($L =
%L_A + L_B$). Error bars have been included for a selection of the
%results.}}
%\end{figure}

Despite the substantial differences in theory and implementation
between the Green-Kubo and direct methods, little work has been
conducted to quantitatively compare the thermal conductivity
predictions obtained by both methods. One exception is the work of
Schelling \textit{et al.} \cite{schelling2002} In that work, the
thermal conductivity of crystalline silicon at a temperature of 1000
K was predicted using both methods and found to agree to within the
measurement uncertainties. Until my recent work on model
Lennard-Jones superlattices,\cite{landry2008a} however, no such
quantitative comparison had been made for nanostructured materials.
In this section, the results of that comparison are discussed.

The thermal conductivity prediction methods were compared by
predicting the cross-plane thermal conductivity of model
Lennard-Jones superlattices with two layers of equal thickness in
the unit cell. These results are shown in Fig$.$ \ref{F-DM_GK_Comp}.
Error bars corresponding to the measurement uncertainties (estimated
to be $\pm$10\% for the direct method and $\pm$20\% for the
Green-Kubo method) are provided for several points. For all but one
point ($R_m = 5$, $5\times5$), the two sets of predictions are in
agreement and neither method consistently underpredicts or
overpredicts the other. We conclude, therefore, that either the
Green-Kubo method or direct method can be used to predict the
superlattice cross-plane thermal conductivity. We note that the
comparison between the thermal conductivity prediction methods was
only made for the cross-plane direction because it is difficult to
predict the in-plane thermal conductivity using the direct method.
This difficulty is due to the very large simulation cell
cross-sectional area [$A_c$, see Fig. \ref{F-DMsimcell}(a)] that is
required to remove the finite cell size-effect on the in-plane
thermal conductivity.

Even though the Green-Kubo method has several advantages over the
direct method (e.g., it predicts both the in-plane and cross-plane
thermal conductivities from one simulation), we prefer to use the
direct method for the thermal conductivity predictions. This choice
is primarily based on the larger prediction uncertainty associated
with the Green-Kubo method than the direct method. All data reported
in the remainder of this proposal correspond to predictions obtained
using the direct method.

\clearpage

\section{\label{S-validation}Validation of the thermal conductivity prediction methodology}

\subsection{\label{S-validation-samples}Samples}

In order to ensure that the MD-predicted thermal conductivity trends
will be observable in experiment, thermal conductivity predictions
were made using the direct method for a series of samples for which
experimental data is available.\cite{landry2008b} Thermal
conductivity predictions were made for bulk Si, bulk Ge,
Si$_{1-x}$Ge$_x$ alloys, and two types of Si/Si$_{1-x}$Ge$_x$
superlattices. In all of the samples, the atoms are initially
located on diamond lattice sites, and the mass of each atom is
randomly assigned according to the natural isotope abundance for
each species.\cite{CRChandbook} The two types of superlattice
studied are (i) Si/Si$_{0.7}$Ge$_{0.3}$ superlattices with the Si
layer being twice as thick as the alloy layer, and (ii) Si/Ge
superlattices with equal Si and Ge layer thicknesses. The
$32\times16$ Si/Si$_{0.7}$Ge$_{0.3}$ and $24\times24$ Si/Ge
superlattices are shown in Fig$.$ \ref{F-unitcells}. The period
length is provided for both superlattices. Note that the interfaces
between the species are perfect (no species mixing or defects) and
are parallel to the (001) crystallographic plane (i.e., the
$xy$-plane, see Fig. \ref{F-unitcells}). Because MD simulations are
only strictly valid in the classical limit, the thermal conductivity
predictions are made at a temperature of 500 K. Quantum effects are
expected to be negligible at this temperature because it is close to
or greater than the experimental Debye temperatures of 658 K and 376
K for Si and Ge.\cite{capinski1997}

%\begin{figure}[tb]
%\centerline{\epsfig{figure=unitcells.eps}}
%\caption{\label{F-unitcells} \small{One period of the the
%$32\times16$ Si/Si$_{0.7}$Ge$_{0.3}$ and $24\times24$ Si/Ge
%superlattices.}}
%\end{figure}

The thermal conductivity predictions are made using relaxed,
zero-stress samples. The zero-stress simulation cell dimensions are
determined from separate MD simulations run in the $NPT$ (constant
mass, pressure, and temperature) ensemble where the stress in each
direction is controlled independently. For bulk Si, bulk Ge, and the
Si$_{1-x}$Ge$_x$ alloys, we find that the zero-stress lattice
constant, $a$, is well-approximated (to within 0.03\%) by
\begin{equation}
a(x) = 5.441 + 0.227x + 0.002x^2   \hspace{2mm} [\textrm{\AA}].
\label{E-a}
\end{equation}
For the Si/Si$_{1-x}$Ge$_x$ superlattices, the lattice constant that
leads to zero-stress in the in-plane directions (the $x$- and
$y$-directions) lies between the bulk lattice constants for Si and
Si$_{1-x}$Ge$_x$, leading to a positive in-plane strain in Si and a
negative in-plane strain in Si$_{1-x}$Ge$_x$. For example, the
zero-stress in-plane lattice constants for the $32\times16$
Si/Si$_{0.7}$Ge$_{0.3}$ and $24\times24$ Si/Ge superlattices are
predicted to be 5.464 \AA \hspace{1mm} and 5.564 \AA \hspace{1mm}
while the bulk lattice constants for Si, Ge, and
Si$_{0.7}$Ge$_{0.3}$ are 5.441 \AA, 5.670 \AA, and 5.509 \AA
\hspace{1mm} [see Eq. (\ref{E-a})]. The zero-stress cross-plane (the
$z$-direction) lattice constants are also different from the bulk
lattice constants due to the in-plane strain. In Fig$.$
\ref{F-layer_spacing}, the predicted layer separations for one
period of the $32\times16$ Si/Si$_{0.7}$Ge$_{0.3}$ and $24\times24$
Si/Ge superlattices are shown. These results are obtained by
averaging the layer separation data over ten different superlattice
periods and 55 ps ($1\times10^5$ time steps). The cross-plane strain
is negative in the Si layer and positive in the Ge and
Si$_{0.7}$Ge$_{0.3}$ layers.

%\begin{figure}[tb]
%\centerline{\epsfig{figure=layer_spacing}}
%\caption{\label{F-layer_spacing} \small{Separation between the
%atomic layers for one period of the the $32\times16$
%Si/Si$_{0.7}$Ge$_{0.3}$ and $24\times24$ Si/Ge superlattices.}}
%\end{figure}

The superlattices modeled are symmetrically strained, meaning that
the in-plane tensile strain in the Si layer balances the in-plane
compressive strain in the Si$_{1-x}$Ge$_x$ layer. In practice,
symmetrically strained superlattices are fabricated by growing (e.g.
with molecular beam epitaxy\cite{huxtablethesis,borca-tasciuc2000})
alternating Si and Si$_{1-x}$Ge$_x$ layers on a substrate that has a
lattice constant equal to the average of the bulk Si and
Si$_{1-x}$Ge$_x$ lattice constants. The advantage that symmetrically
strained superlattices have over non-symmetrically strained
superlattices is that the critical thickness for the formation of
misfit dislocations is greater.\cite{pearsall1999} For example,
symmetrically strained Si/Ge superlattices can be grown with layer
thicknesses up to $\sim$5 nm ($\sim$40 atomic layers) while Si/Ge
superlattices grown on a Si substrate have a maximum layer thickness
of $\sim$1 nm.\cite{pearsall1999}

\subsection{\label{S-validation-k}Thermal conductivity predictions and comparison with experiment}

\subsubsection*{Bulk Si, bulk Ge, and Si$_{1-x}$Ge$_{x}$ alloy}

%\begin{figure}[tb]
%\centerline{\epsfig{figure=designspace_labeled.eps}}
%\caption{\label{F-designspace} \small{Predicted thermal
%conductivities for bulk Si, bulk Ge, the Si$_{1-x}$Ge$_{x}$ alloys,
%and the Si/Si$_{0.7}$Ge$_{0.3}$ and Si/Ge superlattices plotted
%against the fraction of Ge in the unit cell. A second-order
%polynomial fit is included through the alloy data to guide the eye.
%The diffuse and high-scatter limits are provided for comparison.}}
%\end{figure}

The thermal conductivity predictions for bulk Si, bulk Ge, and the
Si$_{1-x}$Ge$_{x}$ alloys at a temperature of 500 K are plotted in
Fig$.$ \ref{F-designspace} against the atomic fraction of Ge in the
unit cell. We predict the thermal conductivities of bulk Si and bulk
Ge to be 103$\pm$21 W/m-K and 61$\pm$12 W/m-K, 35\% and 80\% greater
than the experimental values of 76.2 W/m-K and 33.8
W/m-K.\cite{CRChandbook} For the Si$_{1-x}$Ge$_{x}$ alloys, the
thermal conductivity initially decreases with increasing Ge
concentration until $x \approx 0.3$, beyond which the thermal
conductivity increases with increasing Ge concentration. This trend
is in agreement with experimental measurements at a temperature of
300 K.\cite{maycock1967} To our knowledge, the thermal conductivity
of undoped Si$_{1-x}$Ge$_{x}$ alloys at a temperature of 500 K has
only been been measured experimentally for $x \approx 0.3$ and $x
\approx 0.7$.\cite{abeles1963} At these Ge concentrations, the
MD-predicted thermal conductivities are 50\% and 30\% less than the
experimental measurements of $\sim$5-6 W/m-K.\cite{abeles1963}

\subsubsection*{Diffuse and high-scatter limits}

The thermal conductivity at the diffuse and high-scatter limits is
also provided in Fig$.$ \ref{F-designspace}. The diffuse limit is an
upper bound on the superlattice thermal conductivity. This limit is
reached when (i) the phonon scattering is diffuse within the
superlattice layers (i.e., the layer thicknesses are much greater
than the bulk phonon mean free paths), and (ii) the thermal boundary
resistance at the superlattice interfaces is negligible compared to
the resistance due to conduction within the superlattice layers
(also reached in the limit of large layer thicknesses). The
cross-plane thermal conductivity for a superlattice with two layers
in the unit cell comprised of species $A$ and $B$ at the diffuse
limit, $k_{CP,diff}$, is
\begin{equation}
k_{CP, diff} = \frac{t_A + t_B}{t_Ak_A^{-1} + t_Bk_B^{-1}}
\label{E-kcp},
\end{equation}
where $t_A$ and $t_B$ are the layer thicknesses, and $k_A$ and $k_B$
are bulk thermal conductivities of species $A$ and $B$. The diffuse
limit shown in Fig$.$ \ref{F-designspace} corresponds to that for
Si/Ge superlattices.

The high-scatter limit is a lower bound on the thermal conductivity
and is based on a model proposed by Cahill \textit{et al.}
\cite{cahill1992} This limit is reached when all phonons have a mean
free path equal to half of their wavelength, corresponding to a
material that lacks long range order (e.g., an amorphous phase).
Assuming isotropic and linear phonon dispersion, fully excited
phonon modes (valid for the classical MD system), and harmonic
specific heats, the thermal conductivity at the high-scatter limit,
$k_{HS}$, is
\begin{equation}
k_{HS} = \f{1}{2}\left(\f{\pi}{6}\right)^{1/3} k_{\mathrm{B}}
n_v^{2/3} \sum_i v_i,
\end{equation}
where $n_v$ is the atomic number density and $v_i$ is the sound
speed (i.e., the phonon group velocity in the zero-wavevector limit)
for polarization $i$ (there are two transverse and one longitudinal
polarizations). We estimate $v_i$ for the amorphous phase by scaling
the [100] sound speeds for the Si$_{1-x}$Ge$_x$ alloy by a factor of
0.8 (a typical value for the ratio of the amorphous to crystalline
sound speeds for Si and Ge \cite{cahill1992,holland1963}). The alloy
sound speeds are evaluated using the continuum elastic constants,
which we calculate for the Stillinger-Weber potential using the
analytical approach described by Cowley.\cite{cowley1988} We
estimate the elastic constant dependence on the Ge concentration by
interpolating between the values for bulk Si and bulk Ge, an
approximation that is justified by experimental observations and
theoretical predictions.\cite{baker2000} We note that all of the
MD-predicted thermal conductivities in Fig$.$ \ref{F-designspace}
are above the high-scatter limit.

\subsubsection*{Si/Si$_{0.7}$Ge$_{0.3}$ superlattices}

%\begin{figure}[tb]
%\centerline{\epsfig{figure=alloySLs.eps}}
%\caption{\label{F-Si-alloySLs} \small{Molecular dynamics-predicted
%(body) and experimentally measured\cite{huxtablethesis} (inset)
%thermal conductivities for the Si/Si$_{0.7}$Ge$_{0.3}$
%superlattices. The diffuse limit and the thermal conductivity of an
%alloy with identical Ge composition are provided for
%comparison.\cite{huxtablethesis,dismukes1964}}}
%\end{figure}

The cross-plane thermal conductivity predictions for the
Si/Si$_{0.7}$Ge$_{0.3}$ superlattices are plotted against the
superlattice period length in Fig$.$ \ref{F-Si-alloySLs}. The
experimental thermal conductivity measurements obtained by Huxtable
\textit{et al.} \cite{huxtablethesis} for Si/Si$_{0.7}$Ge$_{0.3}$
superlattices at a temperature of 320 K are provided in the inset of
Fig$.$ \ref{F-Si-alloySLs} for comparison. Huxtable \textit{et al.}
found that the thermal conductivity is nearly independent of
temperature between temperatures of 200 K and 320 K. Therefore, we
can compare our MD predictions to the experimental results at 320 K
by assuming that the thermal conductivity remains
temperature-independent up to a temperature of 500 K.

The predicted thermal conductivities of the Si/Si$_{0.7}$Ge$_{0.3}$
superlattices are between the diffuse limit and the value for an
alloy with the same Ge concentration (i.e., the Si$_{0.9}$Ge$_{0.1}$
alloy), a finding that is in agreement with the experimental
observations. We predict the superlattice thermal conductivity to
decrease with decreasing period length until a period length of 6.55
nm (the $32\times16$ superlattice), beyond which there is a slight
increase in the thermal conductivity. Due to the prediction
uncertainty, however, we cannot conclude that a minimum exists in
the Si/Si$_{0.7}$Ge$_{0.3}$ superlattice thermal conductivity as a
function of period length. We note that thermal conductivity is
experimentally observed to decrease monotonically with decreasing
period length (see the inset in Fig$.$ \ref{F-Si-alloySLs}). This
trend is expected when the phonon transport is incoherent across the
interfaces due to the increasing interface density (see Section
\ref{S-SLs-background}). For the entire period-length range, the MD
predictions are a factor of $\sim$2-3 below the experimental values.


\subsubsection*{Si/Ge superlattices}
%
%\begin{figure}[tb]
%\centerline{\epsfig{figure=Si-GeSLs2.eps}}
%\caption{\label{F-Si-GeSLs} \small{Molecular dynamics-predicted
%(body) and experimentally measured\cite{borca-tasciuc2000} (inset)
%thermal conductivity of the Si/Ge superlattices. The diffuse limit
%and the thermal conductivity of an alloy with identical Ge
%composition are provided for comparison.\cite{lee1997}}}
%\end{figure}

The MD predictions for the cross-plane thermal conductivity of the
Si/Ge superlattices are shown in Fig$.$ \ref{F-Si-GeSLs} along with
the experimental measurements made by Borca-Tasciuc \textit{et
al.}\cite{borca-tasciuc2000} The thermal conductivity of alloys with
identical Ge concentration are also included for
comparison.\cite{lee1997} The MD predictions are made at a
temperature of 500 K while the experimental measurements were made
at a temperature of 300 K. As with the Si/Si$_{0.7}$Ge$_{0.3}$
superlattices, we can compare these predictions and measurements due
to the weak temperature-dependence of the thermal conductivity
observed between temperatures of 200 K and 300
K.\cite{borca-tasciuc2000}

We predict the thermal conductivity of the Si/Ge superlattices to
decrease with increasing period length and then to reach a constant
value of $\sim$12 W/m-K for period lengths greater than or equal to
6.64 nm. The trend of decreasing thermal conductivity with
increasing period length is also observed in the experimental data
of Borca-Tasciuc \textit{et al.} and is indicative of coherent
phonon transport (see Section \ref{S-SLs-background}). The magnitude
of the predicted thermal conductivity, however, is a factor of
$\sim$4 greater than the experimental measurements. A major
qualitative difference is observed between the MD predictions and
experimental measurements as well. We predict the Si/Ge superlattice
thermal conductivities to be greater than that of the
Si$_{0.5}$Ge$_{0.5}$ alloy, while experimentally, the opposite trend
is observed.

\subsection{\label{S-validation-justification}Summary}

The predicted thermal conductivity trends for bulk Si, bulk Ge, the
Si$_{1-x}$Ge$_{x}$ alloys, and the Si/Si$_{0.7}$Ge$_{0.3}$
superlattices are in qualitative agreement with the available
experimental data. This agreement suggests that the MD predicted
thermal conductivity trends will be observable in experiment and
that a mapping function (i.e., a correction factor) can be applied
to the MD data to improve the prediction accuracy. More experimental
data is required to generate such a mapping function. For the Si/Ge
superlattices, the predicted period length dependence is in
agreement with the experimental observations. There is a major
qualitative difference, however, in that the Si/Ge superlattice
thermal conductivities are predicted to be greater than an alloy
with identical Ge concentration while the opposite trend is observed
experimentally. This discrepancy is likely due to the fact that our
superlattices have atomically smooth interfaces and are exactly
periodic. In realistic samples, there is always some degree of
species mixing at the superlattice interfaces. Also, when
superlattices are grown experimentally, there may be some deviation
in the length of each period. Both of these factors may reduce the
ability for phonons to travel coherently through the superlattice
and lead to a lower thermal conductivity. We note that the thermal
conductivity of the Si/Si$_{0.7}$Ge$_{0.3}$ superlattices will not
be as dependent on the sample quality because the phonon transport
is incoherent in these structures. Both effects of species mixing at
the Si/Ge interfaces (see Section
\ref{S-proposed-interfaces-proposed} for details on how species
mixing will be incorporated into the model) and variations in the
period length will be examined in my proposed work.

\clearpage

\section{\label{S-ProposedWork-Interfaces}Proposed research: Quantitatively describe thermal transport across Si/Si$_{1-x}$Ge$_x$ interfaces}

\subsection{\label{S-proposed-interfaces-overview}Overview}

In order to design superlattices with low thermal conductivity, an
understanding of the nature of thermal transport across the
interfaces is required. The current models for thermal boundary
resistance (e.g., the acoustic mismatch and diffuse mismatch
models), however, are not in agreement with experiment at high
temperature and require many assumptions about the nature of the
phonon scattering (e.g., the diffuse mismatch model assumes bulk
material properties and that all phonon scattering is diffuse and
elastic\cite{swartz1989}). In addition, these models do not properly
account for the case of closely-spaced interfaces, where the phonon
properties on either side of the interface cannot be assumed to be
the same as in the bulk materials. \textit{The objective of this
portion of my research is to quantitatively describe the thermal
boundary resistance dependence on (i) alloy composition for isolated
Si/Si$_{1-x}$Ge$_x$ interfaces, (ii) distance between closely-spaced
Si/Ge interfaces, and (iii) degree of species mixing at the isolated
Si/Ge interface.} My preliminary results for the first two items are
discussed in Section \ref{S-proposed-interfaces-preliminary}. The
third item is discussed in Section
\ref{S-proposed-interfaces-proposed} along with additional
equilibrium MD simulations that will be used to gain insight into
the observed trends.

\subsection{\label{S-proposed-interfaces-preliminary}Preliminary results}

\subsubsection*{Thermal boundary resistance dependence on alloy
composition for the isolated Si/Si$_{1-x}$Ge$_x$ interface}

%\begin{figure}[tb]
%\centerline{\epsfig{figure=Si-alloy.eps}}
%\caption{\label{F-Si-alloy} \small{Thermal boundary resistance of an
%isolated Si/Si$_{1-x}$Ge$_x$ interface as a function of the atomic
%fraction of germanium in the alloy layer.}}
%\end{figure}

My preliminary predictions of the thermal boundary resistance of the
isolated Si/Si$_{1-x}$Ge$_x$ interface are shown in Fig.
\ref{F-Si-alloy} as a function of germanium composition in the alloy
layer. The thermal boundary resistance is observed to increase
linearly with increasing germanium concentration in the alloy layer
until $x\approx0.8$, beyond which the thermal boundary resistance
decreases to the value for the isolated Si/Ge interface. For the
isolated Si/Ge interface, the thermal boundary resistance is
predicted to be $2.70 \times 10^{-9}$ m$^2$-K/W, in good agreement
with the value of $2.85 \times 10^{-9}$ m$^2$-K/W predicted by Zhao
and Freund\cite{zhao2005} using a lattice dynamics-based model and
the Stillinger-Weber potential. To our knowledge, no predictions of
the thermal boundary resistance for Si/Si$_{1-x}$Ge$_x$ interfaces
with $x \neq 1$ have been reported.

The thermal boundary resistance trend is different than that
predicted by the acoustic mismatch and diffuse mismatch models,
which predict the thermal boundary resistance to increase
monotonically with increasing dissimilarity (in terms of acoustic
mismatch or mismatch between the density of states, see Section
\ref{S-Intro-Review}) between the interface materials. Equilibrium
simulations are discussed in Section
\ref{S-proposed-interfaces-proposed} which will provide insight and
a possible explanation for this observed trend. The maximum of the
thermal boundary resistance as a function of the alloy composition
suggests that there is an optimum alloy composition that can be used
to minimize the thermal conductivity of Si/Si$_{1-x}$Ge$_x$
superlattices (discussed in Section \ref{S-ProposedWorkSLs}). The
location of this maximum will be more precisely resolved in my
future work by making additional predictions for germanium
concentrations between $0.8 < x < 1.0$. In addition, predictions of
the thermal boundary resistance will be made for several isolated
Si$_{1-y}$Ge$_y$/Si$_{1-x}$Ge$_x$ interfaces. From the results of
these simulations, the effect of acoustic mismatch on the thermal
boundary resistance will be examined by comparing the thermal
boundary resistances of Si/Si$_{1-x}$Ge$_x$ and
Si$_{1-y}$Ge$_y$/Si$_{1-x}$Ge$_x$ interfaces with identical acoustic
mismatch.

\subsubsection*{Thermal boundary resistance dependence on distance
between closely-spaced Si/Ge interfaces}

%\begin{figure}[tb]
%\centerline{\epsfig{figure=sandwich.eps}}
%\caption{\label{F-sandwich} \small{(a) Total thermal resistance (sum
%of the thermal boundary resistances for the two Si/Ge interfaces and
%the conduction resistance of the middle layer) of the Si/Ge/Si and
%Ge/Si/Ge structures as a function of the thickness of the middle
%layer. (b) Example temperature profiles in the interface region for
%a Si/Ge/Si (top) and Ge/Si/Ge (bottom) structure. (c) Schematic of a
%simulation cell containing three closely-spaced interfaces.}}
%\end{figure}

In order to examine the effect of distance between closely-spaced
Si/Ge interfaces, two types of simulation cells were initially
considered. In one case, the simulation cell consists of a germanium
layer sandwiched between two large extents of silicon. In the
opposite case, a silicon layer is placed between two large extents
of germanium. The total thermal resistance, $R_{tot}$, of these
Si/Ge/Si and Ge/Si/Ge structures is plotted in Fig.
\ref{F-sandwich}(a) as a function of the middle layer thickness. The
total thermal resistance is the sum of the thermal boundary
resistances for both interfaces and the conduction resistance of the
middle layer. The diffuse limit (i.e., the total resistance of two
isolated Si/Ge interfaces plus the conduction resistance of the
middle layer assuming bulk thermal conductivities) is also provided
for comparison. For both structures, the total thermal resistance
increases with increasing thickness of the middle layer. The thermal
resistance of the Ge/Si/Ge structure, however, increases at a slower
rate than the Si/Ge/Si structure. This behavior is not yet
understood.

Differences between the Si/Ge/Si and Ge/Si/Ge structures are also
observed in the temperature profiles. The temperature profiles near
the interface region are provided for the Si/Ge/Si and Ge/Si/Ge
structures with middle layer thickness of 28.8 nm and 26.6 nm (a
thickness of 50 conventional diamond unit cells) in Fig.
\ref{F-sandwich}(b). In the Si/Ge/Si structure, approximately 17\%
of the total thermal resistance is due to the conduction resistance
in the Ge middle layer. In the Ge/Si/Ge structure, the conduction
resistance of the Si middle layer is essentially zero, suggesting
ballistic phonon transport. The phonon transport is expected to have
a ballistic component in these structures because the thickness of
the middle layer is smaller than the bulk phonon mean free path
(estimated to be $\sim$50 nm at a temperature of 500 K for both bulk
Si and bulk Ge using the kinetic theory expression for thermal
conductivity). In order to further explore the effect of interface
spacing on the thermal boundary resistance, additional direct method
simulations will be run using a simulation cell like that shown
schematically in Fig. \ref{F-sandwich}(c). This simulation cell will
contain three closely-spaced interfaces. The predicted thermal
boundary resistance dependence on interface spacing will be useful
in determining if there is an optimum interface spacing to maximize
the thermal resistance per unit length and therefore, minimize the
superlattice thermal conductivity.

\subsection{\label{S-proposed-interfaces-proposed}Proposed work}

\subsubsection*{Thermal boundary resistance dependence on species
mixing for the isolated Si/Ge interface}

In realistic interfaces, species can diffuse from one side of the
interface to the other side, forming a thin alloy region near the
interface. The effect of species mixing at the interface on the
thermal boundary resistance has previously been predicted for model
interface systems.\cite{stevens2007,kechrakos1991,twu2003,daly2002}
The results, however, are conflicting. Some
authors\cite{stevens2007,kechrakos1991} have predicted that species
mixing decreases the thermal boundary resistance while
others\cite{twu2003,daly2002} have predicted the opposite trend. In
order to determine the effect of species mixing on thermal transport
across the isolated Si/Ge interface, the thermal boundary resistance
will be predicted for a series of interfaces with varying thickness
of the species mixing region, $D$. The species profile near the
interface will be modeled as
\begin{equation}
x(z) = \frac{1}{2}\left[1 + \tanh\left(\frac{2z}{D}\right)\right],
\label{E-x}
\end{equation}
where $x$ is the germanium concentration and $z$ is the distance
from the interface. Equation (\ref{E-x}) was chosen because it
smoothly transitions between the correct asymptotic limits of $x =
0$ (pure silicon) as $z \rightarrow -\infty$ and $x = 1$ (pure
germanium) as $z \rightarrow \infty$ over a distance approximately
equal to $D$. The thermal boundary resistance will be predicted for
interface thicknesses of one to ten monolayers, which is on the
order of the species mixing region observed in real Si/Ge
superlattices.\cite{soo2001}

\subsubsection*{Identifying the extent of the interface region and
determining the role of phonon coherence in thermal transport across
interfaces}

Equilibrium MD simulations will be used to identify the extent of
the interface region and determine the role of phonon coherence in
thermal transport across interfaces. From these simulations,
important differences in thermal transport behavior between the
various types of interfaces considered in my research will be
identified. My goal is to use these differences to gain insight to
explain the observed thermal boundary resistance trends.

%\begin{figure}[tb]
%\centerline{\epsfig{figure=dos.eps}} \caption{\label{F-dos}
%\small{Molecular dynamics-predicted phonon density of states for
%silicon atoms at the interface and in the bulk region far from the
%interface.}}
%\end{figure}

We define the interface extent to be the region surrounding the
interface where the phonon density of states is different than that
in the bulk regions far from the interface. The phonon density of
states will be calculated from the MD simulations by taking the
Fourier transform of the velocity autocorrelation function. This
calculation can be performed for any atom in the simulation cell,
and therefore, the phonon density of states can be determined as a
function of distance from the interface. For example, the phonon
density of states of a silicon atom located immediately adjacent to
the isolated Si/Ge interface is compared to that for a silicon atom
in the bulk region far from the interface is shown in Fig.
\ref{F-dos}. It is clear that the interface strongly modifies the
phonon density of states. These calculations will be performed for
isolated Si/Si$_{1-x}$Ge$_x$ interfaces with no interfacial species
mixing and isolated Si/Ge interfaces with varying degrees of species
mixing. These calculations will also be performed in the center of
the middle layer in the Si/Ge/Si and Ge/Si/Ge closely-spaced
interface structures.

The role of phonon coherence in thermal transport across interfaces
will be explored using equilibrium MD simulations by examining
correlations of the planar heat flux. The planar heat flux can be
calculated for any plane in the simulation cell and is defined as
the instantaneous flow of heat across that plane. For a system at
equilibrium, the planar heat flux fluctuates about zero. My
hypothesis is that by taking the cross-correlation of these
fluctuations for planes located on either side of the interface, the
role of phonon coherence can be examined. I anticipate that
interfaces with large values of the thermal boundary resistance will
have lower correlation than interfaces with small values of the
thermal boundary resistance. I also suspect that the
Si/Si$_{1-x}$Ge$_x$ interfaces with $x < 1$ will have less
correlation than the Si/Ge interface due to the lack of well defined
phonon modes in the alloy layer (phonons can only exist in an
ordered crystal). Therefore, these calculations may lead to an
explanation for the observed maximum in the thermal boundary
resistance of isolated Si/Si$_{1-x}$Ge$_x$ interfaces. The
correlation between the planar heat flux will also be examined for
points on either side of the middle layer in the Si/Ge/Si and
Ge/Si/Ge interface structures. From these calculations, the level of
phonon coherence across the middle layer can be examined,
potentially leading to an explanation for the observed differences
between the Si/Ge/Si and Ge/Si/Ge structures.

\clearpage

\section{\label{S-ProposedWorkSLs}Proposed research: Designing Si/Si$_{1-x}$Ge$_x$ superlattices for low cross-plane thermal conductivity}

\subsection{\label{S-SLs-background}Background}

Previous experimental and theoretical studies have demonstrated that
superlattices can have very low cross-plane thermal conductivity, a
property desirable for thermoelectric energy conversion. To date,
only a limited number of superlattice designs have been considered
experimentally and theoretically. My hypothesis is that the
previously examined superlattices are not the optimal designs to
minimize the thermal conductivity. Therefore, the current
perspective of the thermoelectric performance of superlattices may
be far from optimum. \textit{The objective of this portion of my
proposed research is to explore the effect of unit cell design on
the thermal transport in superlattices in order to minimize the
cross-plane thermal conductivity of Si/Si$_{1-x}$Ge$_x$
superlattices.} Using a combination of MD simulations, harmonic and
anharmonic lattice dynamics calculations, and the Boltzmann
transport equation, three strategies will be employed to develop
superlattices with low thermal conductivity. These design strategies
are discussed in Section \ref{S-SLs-strategies}.

\subsection{\label{S-SLs-strategies}Strategies for reducing the superlattice thermal conductivity}

\subsubsection*{Strategy 1. Optimize the design of
superlattices with two layers in the unit cell}

%\begin{figure}[tb]
%\centerline{\epsfig{figure=LAeqLB_CPonly.eps}}
%\caption{\label{F-LAeqLB} \small{Cross-plane thermal conductivity of
%the model $L_A = L_B$ Lennard-Jones superlattices plotted as a
%function of total period length ($L = L_A + L_B$). The thermal
%conductivity at the diffuse and alloy limits are also provided for
%comparison.}}
%\end{figure}

The first design strategy is to optimize the design (e.g., alloy
composition and layer thicknesses) of superlattices with two layers
in the unit cell for superlattices in the incoherent phonon
transport regime. In this regime, phonons scatter at the
superlattice interfaces. Therefore, the phonon populations are
distinct within each layer. The cross-plane thermal conductivity of
superlattices with two layers in the unit cell can be predicted in
this regime by adding the thermal resistance of the superlattice
layers and interfaces in series, i.e.,
\begin{equation}
k_{CP} = \frac{t_A + t_B}{(t_Ak_A^{-1} + t_Bk_B^{-1}) + 2R_{A|B}}
\label{E-kcpD},
\end{equation}
where $R_{A|B}$ is the thermal resistance of the $A|B$ interface,
and $t_A$, $t_B$, $k_A$, and $k_B$ are the thicknesses and thermal
conductivities of species $A$ and $B$. For superlattiecs in the
incoherent regime with two layers in the unit cell, Eq.
(\ref{E-kcpD}) predicts the cross-plane thermal conductivity to
decrease with decreasing period length. This trend was observed in
my preliminary study of model $R_m = 2$, $L_A = L_B$ Lennard-Jones
superlattices for superlattices with period lengths greater than
four monolayers (see Fig$.$ \ref{F-LAeqLB}). These results are
presented in Fig$.$ \ref{F-LAeqLB}.

In this design strategy, I will use the predictions of the thermal
boundary resistance (proposed in Section
\ref{S-proposed-interfaces-preliminary}) and thermal conductivity
(described in Section \ref{S-validation-k}) to select the
superlattice species and layer thicknesses that minimize the
cross-plane thermal conductivity given by Eq. (\ref{E-kcpD}). As
presented in Figs. \ref{F-designspace} and \ref{F-Si-alloy}, the
thermal conductivity of Si$_{1-x}$Ge$_x$ and the thermal boundary
resistance of the isolated Si/Si$_{1-x}$Ge$_x$ interface are
functions of the germanium concentration. The thermal boundary
resistance is also a function of the layer spacing [see Fig.
\ref{F-sandwich}(a)]. There is also a constraint on the layer
thickness because the phonon transport will eventually transition
from incoherent to coherent when the phonon mean free path exceeds
the superlattice period length (this regime is the focus of Strategy
2). The thermal conductivity of the optimized superlattices will
then be predicted using MD simulations to examine the validity of
the incoherent assumption and Eq. (\ref{E-kcpD}).

\subsubsection*{Strategy 2. Reduce the phonon group velocities and relaxation
times using complex layering configurations}

The second design strategy is to use complex layering configurations
(i.e., unit cells containing more than two layers) to optimize the
phonon dispersion and scattering properties in the regime where
phonon transport is coherent. The phonon transport is coherent when
the bulk phonon mean free path is greater than the superlattice
period length. In this regime, the phonon scattering is diffusive
(i.e., phonons scatter with other phonons), and the thermal
conductivity can be written as
\begin{equation}
k_{CP} = \sum_i c_{v,i} v_{CP,i}^2 \tau_i, \label{E-kmodes}
\end{equation}
where $c_{v,i}$, $v_{CP,i}$, and $\tau_i$ are the specific heat,
group velocity in the cross-plane direction, and relaxation time for
phonon mode $i$, and the summation is over all available phonon
modes. In the classical limit (realized at all temperatures in MD
simulations and above the Debye temperature in experiment), every
phonon mode is fully excited and the $c_{v,i}$ can be assumed to be
phonon mode-independent. Therefore, the thermal conductivity can be
reduced by finding superlattice designs that have low phonon group
velocities and relaxation times. For superlattices in the coherent
regime with two layers in the unit cell, the average phonon group
velocity in the cross-plane direction has been predicted to increase
with decreasing period
length,\cite{kiselev2000,tamura1999,yang2001,bies2000,broido2004}
leading to a thermal conductivity that increases with decreasing
period length. This trend was observed for the model Lennard-Jones
superlattices for period lengths less than or equal to four
monolayers [see Fig$.$ \ref{F-LAeqLB}].

%\begin{figure}[tb]
%\centerline{\epsfig{figure=dispersion.eps}}
%\caption{\label{F-dispersion} \small{Phonon dispersion curves in the
%[001] direction (see Fig. \ref{F-unitcells} for the definition of
%the coordinate system) for (a) bulk Si and (b) a $2\times2$ Si/Ge
%superlattice calculated from harmonic lattice dynamics.}}
%\end{figure}

A three step process will be used to quickly search the superlattice
unit cell design space for structures that have low phonon group
velocities and relaxation times. The search will be limited to Si/Ge
superlattices with period lengths less than 5 nm because the phonon
transport is believed to be coherent within these structures (see
Fig. \ref{F-Si-GeSLs}). The first step will be to use lattice
dynamics calculations under the harmonic approximation to identify
superlattices with complex layering configurations that have low
phonon group velocities. In the harmonic approximation, a Taylor
series expansion of the system potential energy about its minimum
value is truncated after the second-order
term.\cite{dove,mcgaughey2006} This approach leads to harmonic
equations of motion that can be solved numerically (by generating
and diagonalizing the dynamical matrix) with calculations taking on
the order of seconds. The primary drawback of this approximation,
however, is that no information about the phonon scattering (e.g.,
the phonon relaxation times) can be obtained because the phonon
modes are decoupled. In this step, the superlattice designs will be
generated using genetic algorithms developed by our
collaborator,\cite{landry2008a,mcgaughey2006} Professor Mahmoud
Hussein at the University of Colorado at Boulder. Professor Hussein
has previously shown that genetic algorithms can be used to optimize
the dispersion properties in phononic crystals\cite{hussein2006}. We
believe that these algorithms will allow us to efficiently search
the supperlattice unit cell design space.

The second step will be to use anharmonic lattice dynamics
calculations to predict the phonon relaxation times and the thermal
conductivity for the structures already determined to have low
phonon group velocities. Anharmonic lattice dynamics calculations
introduce the third- and fourth-order terms in the Taylor series
expansion of the system potential energy as a perturbation to the
harmonic solution.\cite{turney2008} These calculations will require
several hours of computation time. If the predicted thermal
conductivity is low (compared to other structures encountered in the
search of the design space), the superlattice design will be added
to a list of structures that will be further explored in step three.
We note that one advantage of using anharmonic lattice dynamics for
the thermal conductivity prediction is that the contribution of each
phonon mode to the total thermal conductivity can be
quantified.\cite{turney2008} This quantification will be valuable in
assessing the validity of the widely invoked assumption that the
acoustic phonons dominate thermal energy
transport.\cite{chen1998b,zeng2001,chen2003,ziambaras2006}

In the third step, the thermal conductivities of the superlattices
predicted by anharmonic lattice dynamics to have low thermal
conductivity will also be predicted using MD simulations. Molecular
dynamics simulations will only be used in this final step of the
design strategy because they are far more computationally expensive
than the lattice dynamics approaches, taking on the order of one
thousand processor hours to make one prediction. This step is
necessary, however, to check the initial assumption of coherent
phonon transport, which will be validated if the anharmonic lattice
dynamics calculation- and MD-predicted thermal conductivities are in
agreement. We have previously demonstrated that these thermal
conductivity prediction methods are in agreement to within the
prediction uncertainty for the monatomic Lennard-Jones
crystal.\cite{turney2008}

The computer codes required to make the harmonic and anharmonic
lattice dynamics calculations proposed in this design strategy are
available within my research group. I have developed a harmonic
lattice dynamics computer code and have experience in making the
calculations and interpreting the results for a variety of solids
described by the Lennard-Jones and Stillinger-Weber
potentials.\cite{landry2008a,mcgaughey2006} The phonon dispersion
curves in the [001] direction [see coordinate system definition in
Fig. (\ref{F-unitcells})] for bulk silicon and a $2\times2$ Si/Ge
superlattice are shown in Fig. \ref{F-dispersion} as an example of
the types of properties that can be obtained from my code. Note that
the superlattice dispersion curves are flatter than the those for
bulk silicon, indicating reduced phonon group velocities (the phonon
group velocity is equal to the derivative of the phonon angular
frequency with respect to the phonon wavevector, $\kappa$). For the
anharmonic lattice dynamics calculations, I will work in
collaboration with Joe Turney, a graduate student in my research
group, who has developed a suitable code as part of his Ph.D. work.

\subsubsection*{Strategy 3. Disrupt the coherent phonon transport in
small period superlattices with two layers in the unit cell}

%\begin{figure}[tb]
%\centerline{\epsfig{figure=disrupted_cell.eps}}
%\caption{\label{F-disrupted_cell} \small{Schematic of the
%$(2\times2)_4(1\times1)$ superlattice unit cell that was found in
%the preliminary Lennard-Jones study to be effective in reducing the
%thermal conductivity below the minimum observed for the $L_A = L_B$
%superlattices (see Fig. \ref{F-LAeqLB}).\cite{landry2008a}}}
%\end{figure}

The transition between the incoherent and coherent phonon transport
regimes produces a minimum in the cross-plane thermal conductivity
as a function of period length for superlattices with two layers in
the unit
cell.\cite{chakraborty2003,venkata2000,caylor2005,simkin2000,chen2005,daly2002,landry2008a}
For the model $R_m = 2$, $L_A = L_B$ Lennard-Jones superlattices,
the minimum was observed for the $2\times2$ structure (see Fig.
\ref{F-LAeqLB}). The third design strategy consists of adding
disruptions to the unit cell of superlattices in which the phonon
transport is coherent.

This design strategy was developed during my preliminary work on
model Lennard-Jones superlattices.\cite{landry2008a} The unit cell
of the $(2\times2)_4\times(1\times1)$ Lennard-Jones superlattice is
shown in Fig. \ref{F-disrupted_cell} as an example. The thermal
conductivity of this disrupted superlattice was predicted to be
0.163 W/m-K, a value that is in the vicinity of the alloy limit
(0.162 W/m-K) and $\sim$20\% less than that of the minimum observed
for the $L_A = L_B$ superlattices (the $2\times2$ superlattice, see
Fig. \ref{F-LAeqLB}). The reduced thermal conductivity observed for
this structure is attributed to reductions in both the phonon group
velocities and the phonon mean free paths in the regime where the
phonon transport has both coherent and incoherent qualities. Because
the phonon transport is coherent in the $2\times2$ superlattice, the
phonon group velocity is reduced compared to that of the bulk
materials.\cite{kiselev2000,tamura1999,yang2001,bies2000,broido2004}
The thermal conductivity is then further reduced by adding
disruptions to these structures. These disruptions decrease the
phonon mean free path below the value that would exist for the
normal $2\times2$ superlattice while maintaining the low group
velocity associated with this structure.

In extending this concept of a disrupted superlattice unit cell to
the design of Si/Si$_{1-x}$Ge$_x$ superlattices, MD simulations will
be the primary prediction tool because they naturally incorporate
the combined effects of incoherent and coherent phonon transport.
The procedure will begin by selecting a Si/Ge superlattice with two
layers in the unit cell that is in the coherent transport regime. My
preliminary calculations (presented in Section \ref{S-validation})
and experimental data suggests an appropriate choice is a
$16\times16$ Si/Ge superlattice. I will examine the effects of the
spacing between the disruptions and the design of the disruptions on
the thermal conductivity. My hypothesis is that in order to have a
low thermal conductivity with the disrupted unit cell design, the
disruption spacing should be large enough to contain several
$16\times16$ subcells in order to recover the low phonon group
velocities associated with the $16\times16$ superlattice. In
addition, the disruption spacing should be on the order of but not
less than the phonon mean free path in order to maximize the effect
of phonon scattering at the disruptions.

\clearpage

\section{\label{S-summaryschedule}Outcomes and schedule}

The outcomes of this proposed research will be:

\begin{itemize}

\item A prediction of the thermal boundary resistance
dependence on (i) alloy composition for isolated Si/Si$_{1-x}$Ge$_x$
interfaces, (ii) distance between closely-spaced Si/Ge interfaces,
and (iii) degree of species mixing at the isolated Si/Ge interface.
These are all properties that are not accessible in experiment.

\item A description of how the interface extent and coherence of the
planar heat flux are influenced by species mixing and alloy
composition of the isolated Si/Si$_{1-x}$Ge$_x$ interface.

\item An assessment of the widely invoked assumption in current
thermal conductivity models that the acoustic phonons dominate the
thermal energy transport.

\item A quantification of how species mixing at the
superlattice interfaces and deviations in the period length affects
phonon coherence and the thermal conductivity of Si/Ge
superlattices.

\item An understanding of the metrics required to design superlattices for low
cross-plane thermal conductivity.

\item A list of Si/Si$_{1-x}$Ge$_x$ superlattice designs predicted to have low cross-plane thermal
conductivity that can be fabricated and experimentally
characterized.

\end{itemize}

The schedule for my proposed research is provided in Fig.
\ref{fig:timeline2}.

\vspace{1cm}

%\begin{figure}[h]
%\centerline{\epsfig{figure=timeline2.eps}}
%\caption{\label{fig:timeline2} \small{Research timeline.}}
%\end{figure}

\clearpage

\section{Biographical Sketch} Eric Landry was born and
raised in Meriden, CT.  He obtained his Bachelor's of Science degree
in mechanical engineering from the University of Connecticut in
Spring 2005. He entered Carnegie Mellon in Fall 2005 and received
his Master's of Science degree in Spring 2007.

\section*{\label{S-awards}Awards}

National Science Foundation Graduate Research Fellowship,
2006-present

First place UConn mechanical engineering senior design project, 2005

UConn Honor's Program Sophomore Certificate, 2003

Dominion Nuclear Scholarship, 2003-2005

Francis T. Maloney Scholarship, 2001-2005

Connecticut Innovations Scholarship, 2001-2005

University of Connecticut Merit Scholarship, 2001-2005

\section*{Peer-reviewed Journal Publications}

\begin{enumerate}

\item E. S. Landry and A. J. H. McGaughey, ``Molecular dynamics prediction of the thermal conductivity of Si/Si$_{1-x}$Ge$_x$ superlattices," In preparation.

\item J. E. Turney, E. S. Landry, A. J. H. McGaughey, and C. H. Amon, ``Thermal conductivity prediction by anharmonic lattice dynamics calculations and
comparison to molecular dynamics methods," In preparation.

\item E. S. Landry, M. I. Hussein, and A. J. H.
McGaughey, ``Complex superlattice unit cell designs for reduced
thermal conductivity," To appear in \textit{Physical Review B}.

\item E. S. Landry, S. Mikkilineni, M. Paharia, and A. J. H.
McGaughey, ``Nanodroplet Evaporation: A Molecular Dynamics
Investigation," \textit{Journal of Applied Physics} \textbf{102}
124301 (2007). Paper selected to appear in Virtual Journal of
Nanoscale Science \& Technology, January 7, 2008 issue.

\item A. J. H. McGaughey, M. I. Hussein, E. S. Landry, M. Kaviany, and
G. Hulbert, ``Phonon band structure and thermal transport
correlation in a layered diatomic crystal," \textit{Physical Review
B} \textbf{74} 104304-1-12 (2006).

\item A. Chaparro, E. Landry, and B. M. Cetegen, ``Transfer
function characteristics of bluff-body stabilized, conical V-shaped
premixed turbulent propane-air flames," \textit{Combustion and
Flame} \textbf{145} 290-299 (2006).

\end{enumerate}

\section*{Conference Presentations}

\begin{enumerate}

\item E. J. Terrell, E. Landry, A. McGaughey, and C. Fred Higgs III,
``Molecular dynamics simulation of nanoindentation," to be presented
by E. T. at IJTC2008, Miami, FL, October 2008.

\item E. S. Landry, M. I. Hussein, and A. J. H. McGaughey,
``Designing Si/Si$_{1-x}$Ge$_x$ Superlattices with Tailored Thermal
Transport Properties," presented by E. L. at Spring 2008 MRS
Meeting, San Francisco, CA, March 2008.

\item E. S. Landry, T. Matsuura, and A. J. H. McGaughey,
``Molecular Dynamics Predictions of the Thermal Boundary Resistance
of Isolated and Closely-spaced Si/Si$_{1-x}$Ge$_x$ Interfaces,"
poster presented by E. L. at Spring 2008 MRS Meeting, San Francisco,
CA, March 2008.

\item E. S. Landry, M. I. Hussein, and A. J. H. McGaughey,
``Dielectric Nanocomposite Layering Configurations for Thermal
Conductivity Reduction," ASME paper MN2008-47052, presented by M. H.
at MN 2008, Sharm El Sheikh, Egypt, January 2008.

\item J. A. Thomas, M. Paharia, E. S. Landry, G. Lee, A. J. H. McGaughey,
``Atomistic Water Droplet Simulation," presented by J. T. at DFD07
Meeting of the American Physical Society, Salt Lake City, UT,
November 2007.

\item E. S. Landry, M. I. Hussein, and A. J. H. McGaughey,
``Molecular dynamics prediction of the thermal conductivity of
Si/Si$_{1-x}$Ge$_x$ superlattices," ASME paper IMECE2007-43177,
presented by A. M. at IMECE 2007, Seattle, WA, November 2007.

\item E. S. Landry, M. I. Hussein, and A. J. H. McGaughey,
``Molecular dynamics prediction of the thermal conductivity of Si/Ge
superlattices," ASME paper HT2007-32152, presented by E. L. at HT
2007, Vancouver, BC, Canada, July 2007.

\item S. Mikkilineni, E. S. Landry, and A. J. H. McGaughey,
``Subcritical and supercritical nanodroplet evaporation: A molecular
dynamics investigation," ASME paper HT2007-32418, presented by E. L.
at HT 2007, Vancouver, BC, Canada, July 2007.

\item E. S. Landry, S. Mikkilineni, and A. J. H. McGaughey,
``Subcritical and supercritical nanodroplet evaporation: A molecular
dynamics investigation," in \textit{Proceedings of the ILASS
Americas 20th Annual Conference on Liquid Atomization and Spray
Systems}, presented by A. M. in Chicago, IL, May 2007.

\item A. J. H. McGaughey and E. S. Landry,
``Exploration of the superlattice thermal conductivity design
space," presented as an invited talk by A. M. at Spring 2007 MRS
Meeting, San Francisco, CA, April 2007.

\item E. S. Landry and A. J. H. McGaughey,
``Exploration of the superlattice thermal conductivity design
space," poster presented at the AVS Western Pennsylvania Chapter's
Nanoelectronic Devices and Materials Symposium, Pittsburgh, PA,
April 2007.

\item E. S. Landry, M. I. Hussein, and A. J. H. McGaughey,
``Superlattice analysis for tailored thermal transport
characteristics," ASME paper IMECE2006-13673, presented by E. L. at
IMECE 2006, Chicago, IL, November 2006.

\item A. Chaparro, E. Landry, and B. M. Cetegen,
``Transfer function characteristics of bluff-body stabilized,
conical premixed tubulent propane-air flames," in
\textit{Proceedings of the Joint Meeting of the U.S. Sections of the
Combustion Institute}, presented by E. L., Philadelphia, PA, March
2005.

\end{enumerate}

\clearpage

\addcontentsline{toc}{section}{8 \hspace{1mm} References}

\bibliographystyle{alan_long}

\bibliography{references_thesis_proposal}

\end{document}
