%--------------------------------------------------------------------------
\chapter{Conclusion (WORK)}
%--------------------------------------------------------------------------
There are two sections to this introductory chapter.

%--------------------------------------------------------------------------
\section{Contributions}
%--------------------------------------------------------------------------

%--------------------------------------------------------------------------
\subsection{Molecular Dynamics-based Methods for Predicting Vibrational 
Lifetimes}
%--------------------------------------------------------------------------

Clarified the proposed spectral method.  

Related it to the properly-defined 
dynamic structure factor. 

%--------------------------------------------------------------------------
\subsection{Thermal Transport in Alloys and High-scatter 
Limits}
%--------------------------------------------------------------------------

New Understanding of Physics 
Our manuscript provides original insight into the physics of thermal 
transport in disordered lattices (i.e., isotopic solids and alloys). 
Notably:
1. The first rigourous test of the virtual crystal (VC) approximation 
when used in anharmonic lattice dynamics (ALD) calculations. The VC-ALD 
technique has been used in a number of recent papers published in PRB 
and PRL (REFS), but its limits have not been assessed.  In VC-ALD, 
the disorder in the alloy is treated as a perturbation. The perturbative 
disorder model was originally developed to model isotopic solids where 
the disorder is weak [11]. We determined the limits of the VC-ALD 
approach using computationally-inexpensive empirical potentials and 
self- consistently treating the disorder explicitly and as a 
perturbation. We are not aware of any such previous study. Our results 
indicate that while VC-ALD is generally an accurate method for materials 
whose thermal conductivity is dominated by low-frequency vibrational 
modes, care must be taken when modeling alloys with low thermal 
conductivities, where significant underprediction of thermal 
conductivity is likely. 
1. 2. The indentification of important connections between the modeling 
of disordered lattices and amorphous materials [1,2,11,35,86]. We 
demonstrate that the high-scatter limit of thermal diffusivity 
typically used in modeling amorphous materials is directly relevant 
to the modeling of disordered lattices. Application of the VC 
approximation leads to vibrational mode diffusivities that are 
non-physical and the high-scatter limit provides a simple, 
physically-sound approach for correcting these predictions. 
Scientific Interest
The breakdown of the VC-ALD method has gone unnoticed in previous 
computational work because these studies: (i) were limited to the 
VC-ALD method because of computationally-expensive DFT calculations, 
so that validation was not possible [12,18-27], (ii) focused on 
materials where the thermal conductivity is dominated by low-frequency 
vibration modes [12,18-27], and/or (iii) did not always compare their 
predictions with experimental measurements [21,22,26]. 
In our work, we provide a self-consistent study of thermal transport 
in disordered lattices using a set of complementary computational tools 
based in molecular dynamics simulations and lattice dynamics calculations. 
The use of empirical potentials versus computationally-expensive DFT 
calculations allowed us to perform the molecular dynamics simulations 
that were necessary to observe the breakdown of the VC-ALD method. Our 
study includes two test materials that demonstrate the applicability 
and breakdown of the VC-ALD method. The conclusions are of general use 
for the study of any disordered lattice. 
The following calculations that we performed are novel additions to the 
literature:
1) Virtual Crystal + Normal Mode Decomposition. To model disordered 
lattices explicitly, we (and others) used normal mode decomposition 
(NMD) on the fully disordered supercell. This approach is limited in 
that the group velocity cannot be extracted so that thermal conductivity 
cannot be predicted.  The novel contribution of our work is the use of 
NMD to predict the lifetimes of a disordered lattice using VC-NMD, 
where the normal modes of the Virtual Crystal (VC) are used as an 
approximation.
2) Allen-Feldman Theory on Disodered Lattice.To model the disorder 
explicitly, we also use the Allen-Feldman (AF) theory of diffusons. 
This theory has only previously been applied to amorphous phases 
[16,17,35,36,74].  We use the AF theory to show that the lower-limit 
of diffusivity of high-frequency modes in a disordered lattice is the 
high-scatter limit, in contrast to the VC-ALD method, which incorrectly 
predicts that the limiting value is zero. Identification of this 
high-scatter limit of mode diffusivity was essential for identifying 
the breakdown in the VC-ALD method.  The high-scatter limit of 
diffusivity is usually assumed, without theoretical justification, in 
models for disordered and amorphous materials [1,2,80,83]. Our study 
gives self-consistent justification for its use.
3) Structure Factor of Disordered Lattice to Predict Group Velocities. 
We calculated the structure factor for modes in a disordered lattice, 
which has previously only been done for modes in amorphous materials.
(cite PRB articles) The structure factor predictions help us to 
understand that the VC-predicted group velocities are an 
underprediction of the representative velocity scale for mode 
diffusivities in the disordered lattice.  While previous studies 
have attempted to predict the group velocity of modes in disordered 
systems, there is no theoretical justification for the methods used 
[60-62]. The structure factor provides a rigorous manner to estimate 
group velocities and is a significant contribution to understanding 
how to predict the correct velocity scale for mode diffusivities in 
disordered systems.[] 
Work Quality
We present a self-consistent study of the VC approximation using five 
different method (VC-ALD, VC-NMD, Gamma-NMD, AF theory, and Green-Kubo). 
We study the thermal transport of Lennard-Jones argon in three solid 
phases of the materials: perfect crystal, disordered lattice, and 
amorphous phase. By using three phases, we demonstrate the applicability 
of the different methods for predicting the thermal conductivity and 
mode-properties:
1) Molecular Dynamics-based Green-Kubo: suitable for modeling all 
three phases, but does not predict the mode properties.
2) Phonon based VC-ALD and VC-NMD: suitable for the perfect crystal 
and disordered lattices with the high-scatter limit correction.
3) The AF theory of diffusons: suitable for the high-frequency modes 
of the disordered lattice and all modes of the amorphous phase. 
We are unaware of any other study that uses all five of these methods 
self-consistently on the same material system. Our work provides clear 
guidelines for others on what tools are appropriate for different 
solid state systems.
Contribution to the Literature
Due to their low thermal conductivities, alloys are currently an active 
area of research, notably in the thermoelectric energy conversion field. 
The ability to predict alloy thermal conductivity is critical in 
narrowing down a large materials design space. Recent papers 
[e.g., PRL 106, 045901 (2011), PRL 109, 095901 (2012), 
PRB 85, 184303 (2012)] have used the VC-ALD method to make such 
predictions.
We believe this work will make an important contribution to the 
literature because the high-scatter limit adjustment is of interest 
to the study of low-thermal conductivity alloys. Thermoelectric energy 
generation materials, such as PbTe/Se alloys [21,22,49], maximize their 
efficiency by minimizing their thermal conductivity. The search for 
lower thermal conductivity alloys will require the modeling of even 
lower-thermal conductivity alloys, where the high-scatter limit we 
have proposed should be considered.


%--------------------------------------------------------------------------
\subsection{Mean Free Paths of Propagating Modes in Amorphous 
Materials}
%--------------------------------------------------------------------------

Draw on results and theory from the amorphous literature, which 
is extensive. 

Present a clear modeling framework for amorphous materials, which 
can be used for studying a 

%--------------------------------------------------------------------------
\section{Future Work}
%--------------------------------------------------------------------------

%--------------------------------------------------------------------------
\subsection{Large Unit Cell Crystals and Disordered Materials}
%--------------------------------------------------------------------------

ALD expensive

eigsoln as well?

MD only option?

%--------------------------------------------------------------------------
\subsection{Computing Exact Normal Modes of Large Finite Systems}
%--------------------------------------------------------------------------

Standard routines for eigenvalue solutions of the dynamical matrix can 
calculate the vibrational normal modes for systems with less than 
8000 atoms 
in less than 24 hours using current computational resources.(cite) 
These eigenvalue solution routines typically perform on single cpus. 
An oppurtunity exists to extend the solution into the parallel cpu 
dimension. This can be accomplished using the 
\href{http://www.mcs.anl.gov/petsc/}{Portable, Extensible 
Toolkit for Scientific Computation}, which has routines for performing 
eigenvalue solutions. The PETSc pacakge has 
\href{http://wiki.python.org/moin/IntegratingPythonWithOtherLanguages}
{Python bindings} contained in the 
\href{https://code.google.com/p/petsc4py/}{petsc4py} 
pacakage, which allows for easy interface 
with existing packages.  
Here is a tutorial on 
\href{https://github.com/thehackerwithin/PyTrieste/wiki/F2Py}
{using the f2py package}, and a general tutorial on 
\href{http://docs.scipy.org/doc/numpy/user/c-info.python-as-glue.html}
{using Python as ``glue''}. 

The controlling Python script would work as follows...
In particular, the GULP package could be 
compiled as a Python module using the 
\href{https://code.google.com/p/f2py/}{f2py} package. The dynamical 
matrix could then be transferred to the parallel eigenvalue solution 
routines contained in PETSc using petsc4py.


A previous studied investigated the vibrational normal modes 
which has been used in a previous study on LJ argon to compute a 
(cite) 
small subset of the normal modes for a system of 32,000 atoms
\cite{mazzacurati_low-frequency_1996} 
using the 
\href{http://en.wikipedia.org/wiki/Lanczos_algorithm}
{Lanczos algorithm}.\cite{golub_matrix_2012} 
The same 32,000 system was subsquently studied using MD simulations 
and the dynamic structure factor,\cite{ruocco_relaxation_2000} 
which was discussed in Section .

%--------------------------------------------------------------------------
\subsection{Predicting Timescales from very Large-Scale Molecular Dynamics 
Simulations}
%--------------------------------------------------------------------------

MD simulations are computationally efficient.  Systems sizes of nearly 
$10^6$ atoms have been studied in this work, and these were bulk systems 
with equal simulation side lengths in all three spatial dimensions. 
The spectral techniques described in Section benefit from not needing 
the eigenvectors of the normal modes of the simulation supercell to 
perform the mapping of the atomic trajectories. These methods, 
combined with appropriately shaped supercells, could probe the 
timescales of vibrational modes up to wavelengths between 
24 and 100 nm using similar computational resources to those 
used in this work. 
This presents an oppurtunity to compare with recent experimental 
measurements spectral linewidths in the frequency range below 
1 THz which have been recently reported for a-SiO$_2$. 
In the case of a-Si, these experimental measurements 
are lacking.\cite{hondongwa_ultrasonic_2011} 
A unique oppurtunity exists to predict the spectral character 
at low frequency before experiment, which is rare in computational 
modeling.(cite)

The current correlation function, closely related to the spectral and 
mode decompostion methods, can even be used to study the spectral 
character of motions in a fluid.\cite{boom_molecular_1980}  
ne study showed that the 
effective dispersion in liquid SiO$_2$ is nearly identical to 
a-SiO$_2$ for low wavevectors (low-frequency).\cite{horbach_high_2001} 

%--------------------------------------------------------------------------
\subsection{DFT and Thermal Modeling}
%--------------------------------------------------------------------------

AF

CP2K, BigDFT

%--------------------------------------------------------------------------
\subsection{``Wrapper`` Package for Thermal Transport Calculations}
%--------------------------------------------------------------------------

There exist numerous packages for performing calculations necessary for 
predicting thermal transport properties. No one package can perform all 
calculations necessary. In particular, no package exists to predict 
mode-level (bottom-up) and system-level (top-down) thermal transport 
properties.   

\begin{center}
\begingroup
%\squeezetable
\begin{table}
\caption{\label{T:available_codes}
Available packages
}
%\begin{ruledtabular}
\begin{tabular}{llllll}
\hline
Package & Language & Capabilities \\
\hline
GULP & Fortran & mode-level (harmonic) \\
LAMMPS & C++ & system-level, mode-level (limited) \\
phonopy & Python & mode-level (harmonic) \\
\end{tabular}
%\end{ruledtabular}
\end{table}
\endgroup
\end{center}

LAMMPS, for example, contains most methods for predicting the 
system-level thermal conductivity. A package to predict the 
mode-level properties is needed.  Ideally, the mode-level properties 
could be predicted alongside the system-level calculations, as is 
necessary to perform the NMD and spectral techniques described in 
Sections and . 

The Python language is an ideal environment for ''gluing`` together 
the availble codes, and extending their functionality in dynamic ways. 






