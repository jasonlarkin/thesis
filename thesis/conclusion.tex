%--------------------------------------------------------------------------
\chapter{\label{Conclusion}Conclusion}
%--------------------------------------------------------------------------

%--------------------------------------------------------------------------
\section{\label{Overview}Overview and Contributions}
%--------------------------------------------------------------------------

%--------------------------------------------------------------------------
\subsection{\label{Overview:SED} 
Molecular Dynamics-based Methods for Predicting Lifetimes}
%--------------------------------------------------------------------------

In Chapter \ref{Chapter:SED}, two MD-based methods for predicting 
phonon properties and thermal conductivity were compared. 
The $\Phi$ method, which is the NMD method in the frequency-domain, 
was properly derived starting with anharmonic lattice 
dynamics theory (see Appendix \ref{Appendix_A:Derivation}). 
The meaning of the proposed spectral method, $\Phi'$, 
was clarified and related to the dynamic structure factor 
(see Appendix \ref{Appendix_B}). 
While 
the $\Phi'$ method does not accurately predict the mode lifetimes, 
the advantage of the $\Phi'$ versus the $\Phi$ method is that it does 
not require an eigenvalue solution for the mode eigenvectors.  
The dynamic structure factor, closely related to the $\Phi'$ method 
(see Appendix \ref{Appendix_B}), can predict frequency-dependent 
timescales from MD simulations for systems with a larger number of 
atoms than those studied in this work using the NMD method 
(see Section \ref{Future:Timescales}). 

%--------------------------------------------------------------------------
\subsection{\label{Overview:VC}
Thermal Transport in Alloys and the High-scatter Limit}
%--------------------------------------------------------------------------

In Chapter \ref{chapter:vc}, thermal transport in two model alloys 
was investigated. The work provides several original 
insight into the physics of thermal 
transport in disordered lattices (i.e., isotopic solids and alloys). 
The first rigorous test of the virtual crystal (VC) approximation 
was presented. The VC-ALD technique has been used in a number of 
recent studies 
\cite{garg_role_2011,tian_phonon_2012,lindsay_thermal_2012,
li_thermal_2012,luckyanova_coherent_2012}, 
but its limits had not been assessed until this study.  
The limits of the VC-ALD 
approach were determined using computationally-inexpensive empirical 
potentials and self-consistently treating the disorder explicitly 
and as a perturbation. 
The results indicate that while VC-ALD is generally 
an accurate method for materials 
whose thermal conductivity is dominated by low-frequency vibrational 
modes, care must be taken when modeling alloys with low thermal 
conductivities, where significant underprediction of thermal 
conductivity is likely. 

% The high-scatter limit of thermal diffusivity,  
% typically used in modeling amorphous materials, is directly relevant 
% to the modeling of disordered lattices. 
% Application of the VC 
% approximation leads to vibrational mode diffusivities that are 
% non-physical and the high-scatter limit provides a simple, 
% physically-sound approach for correcting these predictions. 

The following calculations that were performed are novel additions 
to the literature:
\begin{itemize}
\item Use of the VC-NMD method to model disordered 
lattices explicitly. The novel contribution is the use of 
NMD to predict the lifetimes of a disordered lattice using 
the normal modes of the virtual crystal 
(see Section \ref{S:From VC Gamma} and Appendix \ref{A:NMD XCORR}).
\item  To model the disorder explicitly, the AF Theory calculations 
were performed on a disordered lattice 
(Section \ref{S:Diffusivities_vc}). 
This theory has only previously been applied to amorphous phases 
\cite{feldman_thermal_1993,feldman_numerical_1999,
shenogin_predicting_2009,he_heat_2011}.
The AF theory predictions showed that the lower-limit 
of diffusivity of high-frequency modes in a disordered lattice is the 
high-scatter limit, in contrast to the VC-NMD and VC-ALD methods, 
which incorrectly 
predict that the limiting value is zero. Identification of this 
high-scatter limit of mode diffusivity was essential for identifying 
the breakdown in the VC methods.  The high-scatter limit of 
diffusivity is usually assumed, without theoretical justification, in 
phenomenological models for disordered and amorphous materials 
\cite{kittel_interpretation_1949,graebner_phonon_1986,
cahill_lattice_1988}. 
This study gives self-consistent justification for its use.
\item Calculation of the structure factor of disordered lattices to 
predict effective dispersion (Section \ref{S:Dispersion}). 
The structure factor was calculated for modes in a model disordered 
lattice, which has previously been calculated for modes in amorphous 
materials 
\cite{biswas_vibrational_1988,feldman_thermal_1993,
allen_diffusons_1999,feldman_numerical_1999,
taraskin_determination_1999,taraskin_propagation_2000,
volz_molecular-dynamics_2000,
gotze_evolution_2000,horbach_high_2001,
martin-mayor_dynamical_2001,feldman_calculations_2002,
ciliberti_brillouin_2003,christie_vibrational_2007,
shintani_universal_2008,wyart_scaling_2010,
beltukov_ioffe-regel_2013,larkin_predicting_2013,
marruzzo_heterogeneous_2013}. 
The structure factor predictions help to 
demonstrate that the VC-predicted group velocities are an 
underprediction of the representative velocity scale for mode 
diffusivities in the disordered lattice.  While previous studies 
have attempted to predict the group velocity of modes in disordered 
systems, there is no theoretical justification for the methods used 
\cite{duda_reducing_2011,donadio_atomistic_2009,
he_heat_2011,he_thermal_2011,he_morphology_2011,hori_phonon_2013}. 
\end{itemize}

By using all four methods discussed in Section \ref{S-SystemMode}, 
a self-consistent study of the VC approximation 
identified important connections between the modeling 
of disordered lattices and amorphous materials. 
By using three phases of LJ argon (perfect crystal, disordered lattice, 
and amorphous phase), the applicability 
of the different methods for predicting the thermal conductivity and 
mode-properties was demonstrated:
\begin{itemize}
\item MD-based GK method: suitable for modeling all 
three phases, but does not predict the mode properties.
\item Phonon-based VC-ALD and VC-NMD: suitable for the perfect crystal 
and disordered lattices with the high-scatter limit correction.
\item The AF theory of diffusons: suitable for the high-frequency modes 
of the disordered lattice and all modes of the amorphous phase of 
LJ argon. 
\end{itemize}

% The breakdown of the VC-ALD method has gone unnoticed in previous 
% computational work because these studies: 
% \begin{itemize}
% \item were limited to the 
% VC-ALD method because of computationally-expensive DFT calculations, 
% so that validation was not possible [12,18-27].
% \item focused on 
% materials where the thermal conductivity is dominated by 
% low-frequency vibration modes [12,18-27]
% \item did not always compare their 
% predictions with experimental measurements.\cite{tian_phonon_2012}
% \end{itemize}

% In our work, we provide a self-consistent study of thermal transport 
% in disordered lattices using a set of complementary computational tools 
% based in molecular dynamics simulations and lattice dynamics calculations. 
% The use of empirical potentials versus computationally-expensive DFT 
% calculations allowed us to perform the molecular dynamics simulations 
% that were necessary to observe the breakdown of the VC-ALD method. Our 
% study includes two test materials that demonstrate the applicability 
% and breakdown of the VC-ALD method. The conclusions are of general use 
% for the study of any disordered lattice. 
% 
% The work provides clear 
% guidelines for others on what tools are appropriate for different 
% solid state systems.

%--------------------------------------------------------------------------
\subsection{\label{Overview:MFP}
Mean Free Paths of Propagating Modes in Amorphous Materials}
%--------------------------------------------------------------------------

In Chapter \ref{chapter:mfp}, a clear theoretical and modeling framework 
for amorphous materials was 
presented, which can form the basis for studying a range of disordered 
materials. 
This modeling framework grew as a natural extension of the work and results 
from Chapter \ref{chapter:vc}. 
The NMD-predicted lifetimes, along with the material's sound speed, 
can be used with the AF theory diffusivities to determine the transition 
from propagating to non-propagating modes 
(Section \ref{S:Diffusivities_mfp}). The challenge is that in 
disordered materials, the group velocities 
are not well-defined and there is no theoretical basis to predict 
them \cite{duda_reducing_2011,donadio_atomistic_2009,
he_heat_2011,he_thermal_2011,he_morphology_2011,hori_phonon_2013}. 
Instead, the mode diffusivities are the fundamental quantities, and 
the predictions from both the NMD and AF theory methods must be considered 
simultaneously. 

The following calculations that were performed are novel additions 
to the literature:
\begin{itemize}
\item Identified the effects of metastability in amorphous materials 
on predicting lifetimes using the NMD method 
(see Section \ref{S:Sample} and Appendix \ref{A:metastability}). 
Metastability is likely to affect the application of the NMD method 
in other ordered and disordered systems with weak atomic bonding 
(see Section \ref{Future:LUC}). 
\item Identified differences in the structural properties of 
a-SiO$_2$ and a-Si that lead to a substantial difference in the 
propagating contributions to thermal conductivity in each. 
\item Predicted the effective dispersion from the static structure factor 
to estimate mode group velocities (Section \ref{S:Structure}). 
While effective dispersions have been predicted from the structure factors 
for models amorphous materials previously, they had not been used to help  
predict the thermal conductivity. The effective dispersions 
justify the use of the sound speed at low frequencies. 
\item Using the justified sound speeds, 
it was demonstrated that the NMD-predicted diffusivities are more 
reliable than those predicted by the AF theory at low frequencies 
(Section \ref{S:Diffusivities_mfp}).
\item By comparing predictions from the NMD, AF, and GK methods, it was 
demonstrated that an $\omega^{-2}$ scaling of the low-frequency mode 
lifetimes best describes the model of bulk amorphous silicon 
(Section \ref{S:Bulk}). Comparisons of the predicted thermal conductivity 
accumulations with experimental measurements demonstrated that further 
experimentation is necessary to resolve the low-frequency scaling of the mode 
lifetimes.  
\end{itemize}

%--------------------------------------------------------------------------
\subsection{\label{S:Predictive}
Predictive Ability versus Computational Cost}
%--------------------------------------------------------------------------

With the results from all of the studies presented in this work, 
a new ranking of the predictive capabilities for the four methods 
discussed in Section \ref{S-SystemMode} is made in Table 
\ref{T-comparison-predictive}. 

The GK method played an important role in verifying the mode properties 
predicted by all methods. In Chapter \ref{Chapter:SED}, 
the GK method provided a common comparison for the $\Phi$ and $\Phi'$ 
methods, which helped to confirm 
the disagreement between the two methods. In Chapter \ref{chapter:vc}, 
the GK method provided a comparison for predictions from the VC-NMD 
and VC-ALD methods, which helped to identify the validity of the 
high-scatter limit of the diffuson mode diffusivities in disordered 
lattices. Finally, in Chapter \ref{chapter:mfp}, the GK method helped 
to confirm the scaling of the low-frequency contribution of the 
finite models of a-Si (Section \ref{S:Bulk}). The GK method will 
be a valuable modeling tool for future work on disordered systems. 

The VC-ALD method was shown to be limited to low frequency modes 
and best suited to high-thermal conductivity materials. High thermal 
conductivity materials are typically dominated by the contribution from 
low-frequency modes that are well-modeled 
by VC-ALD. VC-ALD may not be well-suited for low thermal conductivity 
(full spectrum) materials, where the perturbation theory is not 
valid. The AF theory models accurately the mid- and high-frequency 
modes in disordered materials (Section \ref{S:Diffusivities_vc}), 
but it does not properly model the low-frequency 
modes for disordered lattices. It also does not definitively model 
the low-frequency modes in amorphous materials 
(Section \ref{S:Diffusivities_mfp}). 

The VC-ALD method and AF theory can be supplemented by predictions 
from the NMD method, 
but additional assumptions are also required. 
The VC-NMD method is able to accurately predict the lifetimes 
of all vibrons in disordered lattices 
(Section \ref{S:From VC Gamma}). However, the 
effective group velocities are still assumed to be those of the VC, 
which limits the NMD method's predictive ability 
(Section \ref{S:Thermal Conductivity}). Propagating modes in a-Si 
can be identified definitively by NMD-predicted 
lifetimes (see Section \ref{S:Life}), 
but an assumption about the effective mode 
group velocities must be made (Section \ref{S:Diffusivities_mfp}). 
Clearly, predicting group velocities for modes in disordered materials  
is a major challenge that deserves further investigation 
\cite{duda_reducing_2011,donadio_atomistic_2009,
he_heat_2011,he_thermal_2011,he_morphology_2011,hori_phonon_2013}.

With these findings, the predictive methods are re-ranked in order 
of their capabilities in 
Table \ref{T-comparison-predictive}. The NMD method, while the most 
computationally demanding of the four predictive methods 
(Table \ref{T-comparison}), 
is ranked first in mode- and second in system-level predictive 
capability. The reasons for these rankings are:

\begin{itemize}
\item The NMD method ($\Phi$) is derived correctly from anharmonic lattice 
dynamics theory and accurately predicts the mode 
lifetimes and thermal conductivities compared to the $\Phi'$ method.
\item The VC-NMD method accurately predicts the mode lifetimes for 
disordered lattices compared to the VC-ALD method. This leads to better 
agreement with the GK method, which is the most accurate system-level 
method.
\item The NMD method accurately predicts the low-frequency lifetimes 
for a-Si, while the AF theory predictions have large fluctuations that  
depend on the broadening factor. The scaling from the NMD lifetimes is 
used to extrapolate a bulk thermal conductivity which is in good 
agreement with the system-level GK method. 
\end{itemize}

The ALD and AF theory are considered to be equivalent at predicting 
mode-level and system-level properties because, for disordered lattices, 
the VC-ALD method fails to accurately 
predict the mode lifetimes for high frequencies, 
while the AF theory is not valid for low frequencies. Either 
method could be considered superior depending on whether the material 
being studied is a disordered lattice that is low-frequency dominated 
or full-spectrum. The AF theory is superior if the material is amorphous, 
although there have been ALD predictions\cite{fabian_anharmonic_1996} 
of the mode lifetimes in a-Si that are in good agreement with 
NMD predictions\cite{bickham_calculation_1998,bickham_numerical_1999} 
in the literature. 

%--------------------------------------------------------------------------
\begin{center}
%\squeezetable
\begin{table}
\small
\caption{\label{T-comparison-predictive}Ranking of the predictive 
ability from low to high (left to right) of theoretical techniques 
for mode-level and system-level thermal properties for disordered 
systems.}
%\begin{tabular}{llllll}
\begin{tabular}{p{0.9in}|p{0.9in}|p{0.9in}|p{0.9in}|p{0.9in}|p{0.9in}}
\hline\hline
System-level
&CH Theory 
&ALD 
&AF Theory 
&NMD 
&GK \\ 
\hline
Mode-level
&CH Theory 
&GK
&AF Theory
&ALD
&NMD \\ 
\hline\hline
\end{tabular}
\end{table}
\end{center}
%--------------------------------------------------------------------------

%--------------------------------------------------------------------------
\section{\label{Future}Future Work}
%--------------------------------------------------------------------------

%--------------------------------------------------------------------------
\subsection{\label{Future:LUC}Large Unit Cell Materials}
%--------------------------------------------------------------------------

Large unit cell (LUC) materials are an important class of crystalline 
materials with a wide range of thermal transport applications 
\cite{nolas_effect_1998,nolas_skutterudites:_1999,
mcgaughey_thermal_2004,mcgaughey_phonon_2004,ong_surface_2013}. 
LUCs have an ordered (crystalline) structure, but 
the unit cell of the crystal has a large number 
of distinct atoms. LUCs are effectively disordered 
over length scales on the order of the atomic spacing and their thermal 
conductivities can be as low as a glass \cite{cahill_lower_1992}. 
One key advantage of LUC materials is that they are still ordered 
from the standpoint of electrons, which results in good thermoelectric 
performance 
\cite{he_thermoelectric_2006,yang_effect_2006,wang_thermoelectric_2007}. 

Some LUC materials, such as SiO$_2$-based zeolites, have been 
well-studied \cite{mcgaughey_thermal_2004}. 
Others, such as C$_{60}$\cite{olson_specific_1993} or 
\href{http://en.wikipedia.org/wiki/Phenyl-C61-butyric_acid_methyl_ester}
{PCBM}, are currently being investigated for their thermal properties 
\cite{duda_exceptionally_2013}. 
While experimental measurements of PCBM demonstrates that propagating 
modes contribute negligibly, the mechanisms for 
its exceptionally-low conductivity are still not understood. Modeling 
could provide the necessary insights. 

From a modeling perspective, LUC materials pose a number of challenges,
theoretically and computationally, as compared to small unit cell 
(SUC) materials:

\begin{itemize}
\item Predicting model-level properties  
using ALD is challenging because the computational time 
scales as $n^4$ (Section \ref{S-CompCost}). 

\item LUC are crystalline, but are often organic or 
organic/inorganic hybrid materials. The structure of LUC 
materials is often poly- or quasi-crystalline, with less 
long-range order than SUC materials 
\cite{andersson_thermal_1996,ong_surface_2013}.

\item The presence of weak bonding in organic/inorganic materials
\cite{frigerio_molecular_2012,
casalegno_solvent-free_2013,ong_surface_2013} 
can lead to metastability 
(Appendix \ref{A:metastability}), which  
makes it challenging to perform the NMD method. 

\item MD simulations of LUC materials also face challenges. 
While many LUC materials have complex bonding environments, 
DFT calculations 
are too computationally expensive to perform MD simulations 
to predict thermal properties 
\cite{kresse_ab_1993,koker_thermal_2009,huang_filler-reduced_2010,
huang_ab_2008,luo_molecular_2011,esfarjani_heat_2011,
shiomi_thermal_2011}. Even empirical interatomic 
potentials are often computationally-expensive because of the 
complex bonding terms required 
\cite{jorgensen_transferable_1981,jorgensen_comparison_1983,
jorgensen_development_1996,jensen_halide_2006}.
\end{itemize}

Based on the results in this work, there are several 
modeling strategies that can be used to study LUC materials:

\begin{itemize}
\item Identify signs of propagating modes from experimental 
measurements, if available
\cite{efimov_phonon-defect_2004,duda_exceptionally_2013}. 
\item Based on the results for alloys 
(Section \ref{S:Diffusivities_vc}), the AF 
diffuson theory may have application for LUC materials, particularly 
at high frequencies and for those LUC materials which are only 
quasi-crystalline, such as C$_{60}$ \cite{andersson_thermal_1996}.
\item The high-scatter limit for thermal conductivity 
[Eq. \eqref{EQ:M:k_AF,HS}] 
can be used to establish a plausible lower-bound for LUC 
materials \cite{cahill_lower_1992}. Similarly, the 
high-scatter limit for mode diffusivity can establish 
lower-bounds on the mode-level properties. 
\item Perform calculations using computationally-inexpensive 
classical interatomic potentials to assess if DFT calculations 
are necessary. 
\end{itemize}

% %--------------------------------------------------------------------------
% \begin{figure}
% \begin{center}
% \centering
% \includegraphics[scale=0.4]
% {/home/jason/thesis/thesis/duda_exceptionally_2013_fig4.eps}
% \end{center}
% \caption{\label{F:PCBM} 
% Range of thermal conductivity for crystalline, amorphous and LUC 
% materials. Recent measurements on LUC materials show thermal 
% conductivities much lower than bulk amorphous materials. 
% Reproduced from Ref. \citenum{duda_exceptionally_2013}.
% }
% \end{figure}
% %--------------------------------------------------------------------------
% 
% \clearpage

%--------------------------------------------------------------------------
\subsection{\label{Future:Timescales}
Lifetimes from Larger MD Simulations}
%--------------------------------------------------------------------------

%--------------------------------------------------------------------------
\subsubsection{\label{Future:Timescales:NMD}
Exact Normal Modes}
%--------------------------------------------------------------------------

The NMD method used throughout this work is limited by its computational 
demands, which require a larger number ($\sim$ 100) of parallel processors 
to perform the analysis in a reasonable amount of time 
(less than 24 hours).  
%, see Appendix \ref{A:coding_lang:Execute:Scaling}). 
While the NMD method is trivially-parallelizable over the normal 
modes, the eigenvalue solution of the normal modes themselves 
is more computationally demanding. 
The eigenvalue solutions can be performed in parallel using the suggestions 
given in Section \ref{Future:Package}. 
 
While parallel eigenvalue solution can increase the system sizes 
accessible with NMD, the method is ultimately limited by the 
poor scaling of the run time and memory requirements 
(Section \ref{S-CompCost}). 
Additional computational cost can be saved by computing only 
a small subset of the normal modes for a system. 
A previous study used the 
\href{http://en.wikipedia.org/wiki/Lanczos_algorithm}
{Lanczos algorithm}\cite{golub_matrix_2012} 
to compute a small subset of the normal modes
for a 32,000 atom system \cite{mazzacurati_low-frequency_1996}.  
The same 32,000 system was subsequently studied using MD simulations 
and the dynamic structure factor \cite{ruocco_relaxation_2000}, 
which is discussed in the next section.

%--------------------------------------------------------------------------
\subsubsection{\label{Future:Timescales:DSF}
Dynamic Structure Factor}
%--------------------------------------------------------------------------

MD simulations are computationally efficient.  Systems sizes of nearly 
$10^6$ atoms have been studied in this work, which were bulk systems 
with equal simulation side lengths in all three spatial dimensions. 
The dynamic structure factor, described in Section \ref{Appendix_B}, 
can predict vibrational timescales and does not 
need the eigenvectors of the exact normal modes to perform the mapping 
of the atomic trajectories. This method, 
combined with appropriately shaped supercells, could probe the 
timescales of vibrational modes up to wavelengths between 
24 and 100 nm using similar computational resources to those 
used in this work. 
This presents an opportunity to compare with experimental 
measurements of spectral linewidths at frequencies below 
1 THz, which have been recently reported for a-SiO$_2$
\cite{masciovecchio_evidence_2006,baldi_sound_2010,baldi_high_2011,
baldi_emergence_2013}
but are 
lacking for a-Si \cite{hondongwa_ultrasonic_2011}. 
The current correlation function, closely related to the dynamic 
structure factor \cite{horbach_high_2001}, can even be used to study 
the spectral character of motions in a fluid \cite{boom_molecular_1980}.  

%--------------------------------------------------------------------------
\subsection{\label{Future:Package}
Comprehensive Package for Thermal Transport Calculations}
%--------------------------------------------------------------------------

Four different predictive methods were used in this work (Section 
\ref{S-SystemMode}). 
Packages exist for performing some of the calculations 
necessary for these methods. However, no one package can perform all 
necessary calculations, particularly both the 
mode-level and system-level thermal transport 
properties. 
LAMMPS, for example, contains both the GK and direct methods for 
predicting the 
system-level thermal conductivity. A package to predict the 
mode-level properties is needed.  Ideally, the mode-level properties 
could be predicted alongside the system-level calculations, as is 
necessary to perform the NMD and spectral techniques described in 
Section \ref{S:Subsection_NMD}. 

The Python language is an ideal environment for 
\href{http://docs.scipy.org/doc/numpy/user/c-info.python-as-glue.html}
{``gluing``} together 
the available codes and extending their functionality in dynamic ways. 
For example, while the NMD method is trivially-parallelizable over 
the normal modes, the eigenvalue solution of the normal modes themselves 
is more computationally demanding. 
Standard routines for eigenvalue solutions of the dynamical matrix can 
calculate the exact normal modes for systems up to 8000 atoms 
in less than 24 hours using current computational resources 
\cite{gale_general_2003}. 
These eigenvalue solution routines typically run on single 
processors. 
The eigenvalue solutions can be performed using the 
\href{http://www.mcs.anl.gov/petsc/}{Portable, Extensible 
Toolkit for Scientific Computation} (PETSc), which has routines for 
performing eigenvalue solutions in parallel. The PETSc package has 
\href{http://wiki.python.org/moin/IntegratingPythonWithOtherLanguages}
{Python bindings} contained in the 
\href{https://code.google.com/p/petsc4py/}{petsc4py} 
package, which allows for easy interface with the existing lattice 
dynamics package GULP\cite{gale_general_2003} and MD package LAMMPS 
\cite{plimpton_fast_1995}. LAMMPS already contains a 
\href{http://lammps.sandia.gov/doc/Section_python.html}{Python interface}, 
and such an interface could be created for GULP using the 
\href{https://github.com/thehackerwithin/PyTrieste/wiki/F2Py}
{f2py} package.

% Try MD and AF using better scaling DFT packages such as CP2K and BigDFT. 
% BigDFT is linear scaling, but not until a large system size is reached. 

% Discuss ALD, joe's code, ankit's possible code, codes from Esfarjani 
% Shiomi, Broido, phonopy

% The controlling Python script would work as follows...
% 
% \begin{itemize}
% \item The GULP package could be 
% compiled as a Python module using the 
% \href{https://code.google.com/p/f2py/}{f2py} package. 
% \item The dynamical 
% matrix could then be transferred to the parallel eigenvalue solution 
% routines contained in PETSc using petsc4py.
% \item The atomic trajectories are generated using the LAMMPS 
% package with the liblammps Python module.
% \item The 
% \end{itemize}

% \begin{center}
% \begingroup
% %\squeezetable
% \begin{table}
% \caption{\label{T:available_codes}
% Available packages
% }
% %\begin{ruledtabular}
% \begin{tabular}{llllll}
% \hline
% Package & Language & Methods & Capabilities \\
% \hline
% \href{http://projects.ivec.org/gulp/}{GULP} & Fortran & mode-level (harmonic) \\
% \href{http://lammps.sandia.gov/}{LAMMPS} & C++/Python & system-level, mode-level (limited) \\
% \href{http://phonopy.sourceforge.net/}{phonopy} & Python & mode-level (harmonic) \\
% \href{http://phonopy.sourceforge.net/}{phonopy} & Python & mode-level (harmonic) \\
% \end{tabular}
% %\end{ruledtabular}
% \end{table}
% \endgroup
% \end{center}


% LAMMPS as a model of social interaction, mailing list. 
% num-gulp-citations(year) = 
% [1996 1997 1998 1999 2000 2001 2002 2003 2004 2005 2006 2007 2008 2009 2010 2011 2012 2013] 
% 
% num-gulp-citations-year = 
% [1 1 1 1 15 65 59 68 88 77 92 98 114 116 139 128 139 79*(12/9)] 





