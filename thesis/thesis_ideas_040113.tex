% latex thesis_ideas_040113.tex // dvipdf thesis_ideas_040113.dvi
\documentclass[letterpaper,12pt,amsmath,amssymb]{article}
\usepackage[active]{srcltx}
\usepackage[dvips]{epsfig,geometry}
\usepackage{comment}   % package for block comment
\usepackage{graphicx}
\usepackage{hyperref}
\usepackage{url}
\usepackage{breakurl}
\usepackage{color}
\usepackage{xcolor}
\usepackage{listings}

%\usepackage{caption}
%\DeclareCaptionFont{white}{\color{white}}
%\DeclareCaptionFormat{listing}{\colorbox{gray}{\parbox{\textwidth}{#1#2#3}}}
%\captionsetup[lstlisting]{format=listing,labelfont=white,textfont=white}

\geometry{letterpaper,textwidth=6.5in,textheight=9.0in,left=1in,right=1in,top=1in}

%--------------------------------------------------------------------------
\begin{document}
%--------------------------------------------------------------------------


%--------------------------------------------------------------------------
\subsection{Setting Up Computing Environment}
%--------------------------------------------------------------------------


\href{https://github.com/ankitjainmeiitk}{https://github.com/ankitjainmeiitk}


\href{http://www.wikibooks.org}{Wikibooks home}

\lstset{language=Python}
\begin{lstlisting}[label=some-code,caption=Some Code]
public void here() {
	goes().the().code()
}
\end{lstlisting}

80$\%$ of supercomputers using linux
ubuntu: most popular, well-documented, large-community discussion
OS-supported hardware: lenovo, HP, dell, 
apple OS is an acceptable substitute given that it is linux kernel.

setting up environment: $\$$PATH, 
sudo apt-get install gfortran, which gfortran 
gnu compilers vs intel: gnu freely available
tar -zxvf lammps.tar, make serial, sudo apt-get install openmpi, make openmpi

suggested reading:

Statistical Mechanics/Condensed Matter
MacQuarrie
Ashcroft
Wikipedia

Lattice Dynamics
Dove: harmonic autocorrelation (DOS), harmonic (dispersion), 
thermodynamics, structural phase transition, 
Srivistava: quantum formalism, perturbation theory
peierls:
ziman: BTE, RTA
Wikipedia

Quantum Chemistry
Griffiths Quantum Mechanics: Pauli-Exclusion -> Exp repulsion 
Martin Electron Strucutre: DFT
Wikipedia

Math Theory/Numerical Methods
MacQuarrie
Moin
Wikipedia



Wikipedia == open source


list of useful packages: 
pacakges used during this thesis: 
open-source: lammps, gulp, abinit, qe, vesta, phonopy, ntpy
other: vasp

other packages: dlpoly, siesta, GAMES, cp2k, boltztrap, PHON, 

Useful Linux Commands

\href{https://www.google.com/search?q=linux+sed}
{google: linux sed}


sed, grep, ls, cd, pwd, export/setenv, scp -r, ssh, sudo, nohup, 
vi, cat, which (program), echo, qstat -u -n, qdel, 

Running Linux System Commands in Python
 
python vs other languages: combines strengths of the 3 important:
PERL+java+matlab
1) shell-scripting: controlling the OS, system(cmd), (PERL) 

lmp.in.iseed



2) hi-level organization: object-oriented (module.method(input)) (Java)
3) lin alg + numerical methods: numpy.method, scipy.method (fortran, matlab)
python deficiencies
more about this in the appendix

%--------------------------------------------------------------------------
\subsection{Preparing Journal Articles and Thesis}
%--------------------------------------------------------------------------





%--------------------------------------------------------------------------
\subsection{Coding Languages}
%--------------------------------------------------------------------------

linux/unix shell

\href{http://www.youtube.com/watch?v=XFQ9dw3CyDo&list=PL1D10C030FDCE7CE0}
{excellent c++ tutorial}



\href{http://www.youtube.com/watch?v=YRTEOFMUTzw&list=PL6A8E21D2E86A0155}
{excellent fortran tutorial}

matlab has an excellent built-in guide, google search will typically 
yield useful results. 

\href{http://www.gnu.org/software/octave/}
{matlab}

\href{http://www.gnu.org/software/octave/}
{octave}

%--------------------------------------------------------------------------
\subsection{Online Resources}
%--------------------------------------------------------------------------

wikipedia

\href{http://www.sklogwiki.org/SklogWiki/index.php/Main_Page}
{octave}

https://en.wikipedia.org/wiki/Lennard-Jones_potential

http://www.sklogwiki.org/SklogWiki/index.php/Lennard-Jones_model

we can do better, needs to be organization. 

http://nanohub.org/

does not provide good HPC resources. 

%--------------------------------------------------------------------------
\subsection{Latex Resources}
%--------------------------------------------------------------------------

http://tex.stackexchange.com/questions/1596/how-to-compile-a-latex-document

http://stackoverflow.com/questions/2461905/compiling-latex-bib-source

http://publish.aps.org/revtex

used 

http://kile.sourceforge.net/

http://www.zotero.org/







%--------------------------------------------------------------------------
\subsection{Google Searching}
%--------------------------------------------------------------------------

Web-searching. 

keywords: integrated development environment (IDE), 

%--------------------------------------------------------------------------
\subsection{Symmetry Discussion}
%--------------------------------------------------------------------------

rub abinit for si, get S

$k_{ij} = k_{ji}$  and $k_{ii} = k{jj}$ show using symmetry operations

$S = [-100;010;001 ]$


This applies to any vector or tensor property of the crystal.
      
\cite{sinha_scattering_2005}

\href{https://github.com/jasonlarkin/disorder/blob/master/matlab/lj/phonon/LJ_PRIM_CONV_MAPPING.m}
{lennard-jones FCC primitive conventional mapping}


%--------------------------------------------------------------------------
\subsubsection{KPT symmetries}
%--------------------------------------------------------------------------

property(k) = property(-k)  for all

show how kpts are reduced to first octant using just rotations

show how kpts in first octant are reduced.

%--------------------------------------------------------------------------
\subsection{Normal Mode Decomposition}
%--------------------------------------------------------------------------
harmonic assumption
MD is anharmonic

%--------------------------------------------------------------------------
\subsubsection{Eigvec mapping}
%--------------------------------------------------------------------------

%--------------------------------------------------------------------------
\subsection{VC-NMD versus Gamma-NMD}
%--------------------------------------------------------------------------


%--------------------------------------------------------------------------
\section{Computability}
%--------------------------------------------------------------------------

research.idea = 'string'

%--------------------------------------------------------------------------
\section{Unsorted}
%--------------------------------------------------------------------------
 
the smartest ppl ion the world shud be using the same shit

%--------------------------------------------------------------------------
\section{Perspectives}
%--------------------------------------------------------------------------

%--------------------------------------------------------------------------
\subsection{Preparing Journal Articles}
%--------------------------------------------------------------------------

The job of the student is to prepare, submit, and (hopefully) publish 
peer-reviewed journal articles. There are many journals suitable for 
nanoscale transport topics.  Here is one example of the latex files 
used to create an article used in this work:

This file uses Revtex, which is an article class used by Physical Review, 
Journal of Applied Physics, and others. The paper can be compiled using:



These can be used as templates



Here is a Carnegie Mellon thesis template:

https://github.com/robsimmons/cmu-thesis

Useful article on structuring large documents

http://web.science.mq.edu.au/~rdale/resources/writingnotes/latexstruct.html



1) Use latex
- reccomend Kile editor, gedit works well too. 
- reccomend zotero to maintain library.bib

1) Time to prepare each paper

reccomendation: student advisor should try and exchange editing a written 
research document every day, at least every week.  

Such a research document could be the running 
collection of journal articles which turn into the thesis. Maintenance 
of this document can be achieved with Dropbox or Github.  Github 
offers to advantage of smart version control and internet wiki. 

2) Computational cost of each paper

CPU, memory
parallel cpu, parallel memory

Trends: 
small single cpu jobs -> large parallel cpu jobs
single predictive method -> several predictive methods

reccomendation: explore all predictive methods and models possible

%--------------------------------------------------------------------------
\subsection{Scaling Calculations}
%--------------------------------------------------------------------------

combination of increasing hardware, improving software.  The 
improvement of these aspects depends partly on current 
research.  

Have most control over what calculations

Time-scales for data points: 
1 sec, 1 min, 1 hour, 1 day, 1 week, 1 month, 1 year 
number of calculations per PhD:
1E8 , 1E6 , 1E4, 1E3 , 1E2 , 1E1 , 1

1) Run system sizes by doubling. Most calcualtions scale poorly (i.e., 
(SYS_SIZE^alpha with alpha > 1). Pick a maximum system size based 
on computatinal resources and time constraints.  Run all system sizes 
halved before running maximum system size. 

2) Run all trivial (< 5 mins) calculations on a given system using all 
methods possible. 

Explore all trivial calculations first. Increase the time-scale and 
decrease the numner of data-point accordingly. 

reccomendation: 
%--------------------------------------------------------------------------

%--------------------------------------------------------------------------
\subsection{Plottable Code}
%--------------------------------------------------------------------------

It is important to explore various predictive methods visually.  
reccomendation: matplotlib provides an enormous number of plotting 
tools.

Here is a simple example using the python wrapper for lammps 
with matplotlib to visualize. An MD simuation of LJ argon 
is visualized. An atom is placed into an unusually high energy 
position, which initiates melt to the liquid phase. 

matplotlib, code version with and without plotting capability.

%--------------------------------------------------------------------------
\subsection{Scaling into Multi-CPU Dimension}
%--------------------------------------------------------------------------

%--------------------------------------------------------------------------
\subsubsection{Python w/o shared Memory}
%--------------------------------------------------------------------------

%--------------------------------------------------------------------------
\subsubsection{Python w/ shared Memory (Parallel Python)}
%--------------------------------------------------------------------------

%--------------------------------------------------------------------------
\subsubsection{Compiled Code w/ Python Wrapper: Fortran and C++}
%--------------------------------------------------------------------------

cython: example LAMMPS python wrapper
f2py: simple example. suggest warpper be built for GULP.

%--------------------------------------------------------------------------
\section{Importance of Open-Source Code}
%--------------------------------------------------------------------------

video of python development tree

\href{http://www.youtube.com/watch?v=3poNeQHUKrs}
{google+ history with gource}

\href{http://www.youtube.com/watch?v=cNBtDstOTmA}
{python history with gource}

grand-daddy of them all, 
\href{http://www.youtube.com/watch?v=pOSqctHH9vY}
{linux kernel history with gource}

It is important that results become more open-source.  It's important 
that our communication becomes open-source. It's important that the 
entire numerical process be carried out open-source. 

%--------------------------------------------------------------------------
\subsection{Redundancy}
%--------------------------------------------------------------------------

There are 3 different codes (shiomi/esfarjani, broido, ankit) for 
performing the same basic calculations.

For LD, there is GULP, phonopy, SIESTA, etc.

%--------------------------------------------------------------------------
\subsection{Code Development}
%--------------------------------------------------------------------------

took ankit 10 months to re-develop esfarjani code.  

code development time can be drastically reduced using pre-existing code. 
codes written in a modular fashion can be added to easily.

%--------------------------------------------------------------------------
\subsection{Experiment Pre-dating Simulation}
%--------------------------------------------------------------------------
The ideal goal of simulation is to pre-date experiment.  This has not 
been achieved yet.  See Fig.

findings for:
CNT/Graphene
Si
Thermoeletric (LUC, alloys, SL, etc)
Perovskites
PCM

%--------------------------------------------------------------------------
\subsection{Experiment Pre-dating Simulation}
%--------------------------------------------------------------------------

ntpl most cited paper 1-5:
maradudin, ladd, dove, ziman, 






\clearpage
\addcontentsline{toc}{section}{8 \hspace{1mm} References}
\bibliographystyle{plain}
\bibliography{/home/jason/Dropbox/ntpl-paper/ntpl-121612}
\end{document}

