%--------------------------------------------------------------------------
\chapter{Introduction (WORK)}
%--------------------------------------------------------------------------
There are two sections to this introductory chapter.

Crystals well understoopd, except for brand-new stuff(cite broido, 
esfajani)

Disordered materials, what is known or unknown.

Physics community lots of studies, but nothing pushing the towards 
thermal.

Kinetic Theory

Callaway/Holland-type

%--------------------------------------------------------------------------
\section{Thermal Conductivity Theory}
%--------------------------------------------------------------------------

%--------------------------------------------------------------------------
\subsection{Green-Kubo Formulation}
%--------------------------------------------------------------------------

Discussed in other works.



%--------------------------------------------------------------------------
\subsection{Kinetic Phonon Gas and Perturbation Theory}
%--------------------------------------------------------------------------

For a perfect lattice, 
all vibrational modes are phonon modes, which by 
definition are delocalized, propagating plane waves.
\cite{ziman_electrons_2001} Using the single-mode relaxation
time approximation \cite{ziman_electrons_2001} to solve 
the Boltzmann transport equation gives an 
expression for thermal conductivity kph Eq. \ref{EQ:k_vib}. 

The relaxation time approximation has been found to be valid  
for lower thermal conductivity materials 
(e.g., Si and SiGe alloys),
\cite{broido_intrinsic_2007,ward_intrinsic_2010,garg_role_2011} 
while larger thermal conductivity 
materials such as GaN and diamond require an  
iterative solution to the BTE for more accurate predictions 
using Eq. \eqref{EQ:k_vib}.
\cite{ward_ab_2009,lindsay_thermal_2012} 


In the VC-ALD method, the phonon-phonon scattering is  
predicted using ALD.\cite{turney_predicting_2009,esfarjani_heat_2011} 
with phonon-phonon and phonon-defect scattering treated as 
perturbations.
\cite{garg_role_2011,tian_phonon_2012,lindsay_thermal_2012} 
Phonon-phonon scattering in ALD is modeled by including three-phonon 
processes.\cite{turney_predicting_2009-1,garg_role_2011,tian_phonon_2012} 
The present study is concerned with temperatures much less than the 
melting temperature of either LJ argon
\cite{mcgaughey_phonon_2004} or 
SW silicon\cite{stillinger_computer_1985} so that we believe the effects 
of higher-order phonon processes are 
negligible.\cite{ecsedy_thermal_1977,turney_predicting_2009-1} 
We predict the phonon-phonon lifetimes using the method 
described in Ref. \citenum{turney_predicting_2009-1}, 
with all classical expressions for the populations to remain 
consistent with the classical MD-based methods from 
Section \ref{S:From VC Gamma}. 




The phonon-defect scattering is treated 
using perturbative methods that can handle mass and/or bond disorder.
\cite{klemens_scattering_1955,klemens_thermal_1957,mattis_phonon_1957,
tamura_isotope_1983} 
Using perturbation theory, Tamura derived a general expression for 
phonon scattering by mass point defects to second order that was applied 
to study isotopic germanium.\cite{tamura_isotope_1983} Bond disorder 
can be accounted for using a similar expression with an average
atomic radius or suitable scattering cross-section.
\cite{klemens_scattering_1955,klemens_thermal_1957} 
In Ni$_{0.55}$Pd$_{0.45}$, 
which has a large mass ratio (1.8) and concentration of each species, 
experimental measurements of   
vibrational frequencies and linewidths agree well with 
predictions from the perturbative mass-disorder theory.
\cite{mattis_phonon_1957,kamitakahara_vibrations_1974,tamura_isotope_1983} 

Using DFT methods to predict 
the mode-specific phonon properties of the VC, Lindsay and Broido 
found good agreement between VC-ALD and experimental measurements of 
thermal conductivity for 
isotopically defected GaN (the gallium isotopes have concentrations of 
0.6 and 0.4 and a mass ratio of 1.03).\cite{lindsay_thermal_2012} 
Garg et al. used DFT calculations with VC-ALD   
to predict the thermal conductivity of SiGe alloys 
for all concentrations at a temperature of 300 K, 
obtaining good agreement with experiment.\cite{garg_role_2011}  
By including disorder explicitly in their ALD calculations, the predicted 
thermal conductivity decreased by 15$\%$. 
Isotopically-defected GaN and low concentration SiGe alloys 
have relatively large 
thermal conductivities at a temperature of 300 K ($\sim$ 100 W/m-K). 
Li et al. used DFT calculations with VC-ALD to predict the thermal 
conductivity of Mg$_2$Si$_x$Sn$_{1-x}$ ($\sim$ 10 W/m-K) 
in good agreement with 
experimental measurements for all concentrations.\cite{li_thermal_2012}
The VC-ALD approach has also been used to predict the effect of interfacial 
mixing in GaAs/AlAs superlattices, but the thermal 
conductivity predictions were not compared with experimental 
measurements.\cite{luckyanova_coherent_2012}  
In our survey of experimental measurements and numerical modeling, 
we find that 
VC predictions tend to be accurate when the disordered lattice 
thermal conductivity 
is significantly above the high-scatter limit [Eq. \eqref{EQ:M:k_AF,HS}],  
which tends to be around 1 W/m-K.
\cite{abeles_lattice_1963,kamitakahara_vibrations_1974,
cahill_thermal_2004,cahill_thermal_2005,cahill_lattice_1988,
garg_role_2011,lindsay_thermal_2012} 

An ALD study using phonon properties from DFT calculations for 
crystalline PbTe\cite{shiga_microscopic_2012} predicted 
thermal conductivities of 2 W/m-K at a temperature of 300 K 
in fair agreement with experiment. 
For PbTeSe alloys, a VC-ALD 
study predicted a small thermal conductivity reduction compared to the 
perfect crystals.\cite{tian_phonon_2012} Experimental results are limited 
for these alloys,\cite{kudman_thermoelectric_1972,pei_convergence_2011} 
making it difficult to asses the validity of the VC-ALD approach for 
materials whose thermal conductivities approach the high-scatter limit.

Given all these results, it is unclear what limitations exist for 
using the VC approach. 
In this study, we will consider a low thermal conductivity alloy  
using the LJ potential and a high thermal conductivity alloy using the 
SW potential. The computational studies discussed above were 
limited to VC-ALD 
because of DFT calculation costs. Our use of computationally 
inexpensive empirical potentials allows us to include the disorder 
explicitly and as a perturbation and to compare the predictions. 


%--------------------------------------------------------------------------
\subsection{Diffuson Theory}
%--------------------------------------------------------------------------

For disordered systems, the vibrational modes are no 
longer pure plane-waves (i.e., phonon modes), except in the low-frequency 
(long-wavelength) limit. When applied in the classical limit, 
the Allen-Feldman (AF) theory computes 
the contribution of diffusive, non-propagating modes (i.e., diffusons) 
to thermal conductivity.\cite{allen_thermal_1993} 

kAF Eq. \eqref{EQ:M:k_AF} 
The diffusivity of diffusons 
can be calculated from harmonic lattice dynamics theory.
\cite{allen_thermal_1993,feldman_thermal_1993,feldman_numerical_1999} 

Assuming that all vibrational modes travel with the sound speed, $v_s$, and 
scatter over a distance of the lattice constant, $a$, 
a high-scatter (HS) limit of thermal conductivity in the classical 
limit is\cite{cahill_lattice_1988} 
\begin{equation}\label{EQ:M:k_AF,HS}
k_{HS} = \frac{k_{\text{B}}}{V_b}b v_s a,
\end{equation}
where $V_b$ is the volume of the unit cell and $b$ is the number of atoms 
in the unit cell. The HS limit will be used to 
discuss the differences between the LJ argon and SW silicon alloys. 

Harmonic theory

Does not work at low-frequencies

%--------------------------------------------------------------------------
\subsection{Bottom-up versus Top-Down Thermal Conductivity}
%--------------------------------------------------------------------------

Callaway-Holland (CH) theory uses empircally fit models for the 
mode properties. The CH empircal models can be used together 
with harmonic lattice dynamics calculations on atomistic 
models based on interatomic potentials or DFT calcualtions. 

Because empirically fit models are used, the CH theory lacks 
predictive capability. 

Anharmonic lattice dynamic (ALD) can be used to predict 
phonon properties without the need for empircal fitting. 


Allen-Feldman (AF) theory

MD-based Green-Kubo (GK) method

normal mode decomposition (NMD)



\begin{center}
%\squeezetable
\begin{table}
\small
\caption{\label{T-comparison}Qualitative comparison of theoretical 
techniques for predicting vibrational mode properties and thermal 
conductivity.}
%\begin{tabular}{llllll}
\begin{tabular}{p{0.9in}|p{0.9in}|p{0.9in}|p{0.9in}|p{0.9in}|p{0.9in}}
\hline\hline
&CH Theory 
&ALD
&AF Theory
&GK 
&NMD\\ 
\hline\hline
Description
&Empirical
&Predictive 
&Predictive 
&Predictive
&Predictive\\
\hline
Input
&Empirical Models, dispersion and lifetime models, experimental $k$ data 
&Force constants \newline (classical or DFT)  
&Force constants \newline (classical or DFT)
&Forces (classical or DFT) \\
\hline
Temperature
&All 
&Low 
&All
&All
&Mid-range\\
\hline
Disorder
&As additional \newline perturbation 
&As additional \newline perturbation 
&Can be naturally included 
&stuff
&\\
\hline
Anharmonicity
&Full 
&Third-order 
&Full 
&Harmonic 
&Quasi-Harmonic\\
\hline
Statistics
&Bose-Einstein or \newline Boltzmann 
&Bose-Einstein or \newline Boltzmann 
&Boltzmann 
&Bose-Einstein or \newline Boltzmann
&Boltzmann\\
\hline
Mode Range
&Low-frequency 
&All-frequency (crystal), low-frequency (alloy) 
&
&stuff
&\\
\hline \hline
\end{tabular}
\end{table}
\end{center}
%--------------------------------------------------------------------------



Comments

Callaway/Holland model: 
Challenging to apply to non-isotropic systems. 
Requires fitting parameters. 


ALD
Partial fourth-order theory for predicting frequency shifts available. 
Calculation scales with the fourth power of the number of 
atoms in the unit cell. 

Requires $\emph{a priori}$ harmonic lattice dynamics calculations. 
Can also predict $k$ from the mode-level properties.


The Allen-Feldman (AF) theory can use quantum or classical expressions 
for the specific heat, while the mode properties it predicts are 
based on the harmonic approximation. While this should limit the 
applicability of the AF theory to low temperatures where the harmonic 
approximation is valid, anharmonic effects have been shown to be 
small for disordered solids such as amorphous silicon.
\cite{feldman_thermal_1993}

NMD Mid-range\cite{turney_predicting_2009-1}


%--------------------------------------------------------------------------
\subsection{Computational Cost and Modeling Choices}
%--------------------------------------------------------------------------

The key to explicitly incorporating the effects of disorder 
is to use large disordered supercells. 

Efficient MD 
codes like LAMMPS scale linearly with the number of atoms in 
the system, $N_a$, which makes the GK method (see Section 
\ref{S:Thermal Conductivity}) 
computationally-inexpensive when used with empirical potentials. 




