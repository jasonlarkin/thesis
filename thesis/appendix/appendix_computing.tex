%--------------------------------------------------------------------------
\chapter{\label{Appendix_HPC}
Research using High Performance Computing (WORK)}
%--------------------------------------------------------------------------

%--------------------------------------------------------------------------
\section{\label{A:Comp_Env}Setting Up Computing Environment}
%--------------------------------------------------------------------------


%%--------------------------------------------------------------------------
%\subsection{Suggested Reading}
%%--------------------------------------------------------------------------


%The online encyclopedia \href{http://www.wikipedia.org/}{wikipedia} 
%has become an excellent resource for technical knowledge. 

%\href{http://en.wikipedia.org/wiki/Statistical_mechanics}
%{Statistical Mechanics}/
%\href{http://en.wikipedia.org/wiki/Condensed_matter}
%{Condensed Matter}
%\cite{ashcroft_solid_1976,mcquarrie_statistical_2000}

%\href{http://en.wikipedia.org/wiki/Phonon}
%{Lattice Dynamics} with focus on the classical, quantum 
%and thermodynamic properties of phonons.\cite{peierls_,ziman,
%wallace_thermodynamics_1976,
%srivastava_1990,dove_lattice_1993} 

%\href{http://en.wikipedia.org/wiki/Introduction_to_quantum_mechanics}{Introduction to quantum mechanics and quantum chemistry}
%\cite{griffiths_introduction_1995} 
%and more 
%advanced topics on 
%\href{https://en.wikipedia.org/wiki/Electron_configuration}
%{Electron Strucutre} and 
%\href{http://en.wikipedia.org/wiki/Density_functional_theory}
%{Density Functional Theory}.
%\cite{martin_electronic_2004} 

%Analytical methods for 
%\href{http://en.wikipedia.org/wiki/Ordinary_differential_equations}
%{ordinary} and 
%\href{http://en.wikipedia.org/wiki/Partial_differential_equation}
%{partial differential equations}, 
%\href{http://en.wikipedia.org/wiki/Fourier_analysis}
%{fourier} and 
%\href{http://en.wikipedia.org/wiki/Statistics}
%{statistical analysis},\cite{mcquarrie_mathematical_2003} 
%and 
%\href{http://en.wikipedia.org/wiki/Numerical_analysis}
%{numerical analysis}.\cite{moin_fundamentals_2010} For example, 
%the LJ MD code discussed in Section uses a  
%\href{http://en.wikipedia.org/wiki/Verlet_integration}
%{verlet integration method}. 

%--------------------------------------------------------------------------
\subsection{\label{A:Comp_Env:OS}Hardware and Operating System Choice}
%--------------------------------------------------------------------------

The choice of hardware determines the operating system. The three 
main choices for operating system are Windows, Apple OS, and Linux. Each 
operating system has limitations depending on the hardware it 
operates on.   
For example, Apple OS is primarily used on Apple hardware.

\href{http://www.youtube.com/watch?v=7XTHdcmjenI}
{Linux: world's most widely used software}
Linux runs on cell phones to the world's largest supercomputers. 
Reccomend the most widely used linux version 
\href{http://www.ubuntu.com/}{ubuntu}.
well-documented, large-community discussion
Apple OS is an adequate substitute as it is 
\href{https://en.wikipedia.org/wiki/Unix-like}{unix-like}.

There are many options for installing Ubuntu and the instructions 
can be found on \href{http://www.ubuntu.com/}{http://www.ubuntu.com/}.
Ubuntu is certified on 
\href{http://www.ubuntu.com/certification/}
{many top PC's from many computer companies}
and a company, \href{https://www.system76.com/}{system 76}, 
builds computers with Ubuntu pre-installed. I used the 
\href{https://www.system76.com/laptops/model/panp9}
{Pangolin Performance} for over a year of my PhD.  However, 
I reccomend using the lightest, most-portable, and longest-battery 
life notebook available such as the 
\href{https://help.ubuntu.com/community/SamsungSeries9}
{Samsung Series 9}  
or \href{https://help.ubuntu.com/community/MacBookAir}{Macbook Air}. 
You will typically be using this notebook to access large computing 
clusters remotely, 
so there is a benefit of having portability and long 
battery life over large computational power.  

%--------------------------------------------------------------------------
\subsection{\label{A:Comp_Env:Term}
Linux/Unix Terminal, Commands, and Environment Variables}
%--------------------------------------------------------------------------

These instructions work best for Ubuntu operating system, and will work 
well for other versions of Linux. Systems commands are executed by the 
system 
\href{https://help.ubuntu.com/community/UsingTheTerminal}{terminal}. 


Useful Linux Commands:

sed, grep, ls, cd, pwd, export, setenv, scp -r, ssh, sudo, nohup, 
vi, cat, which, echo

\href{https://www.google.com/search?q=linux+sed}
{google: linux sed} 


\href{http://en.wikipedia.org/wiki/Environment_variable}
{environment variable} such as $\$$PATH.  Here is how we determine what 
PATH is set to:


Set PATH for lammps (see Section ).


Permanent changes to environment variables can be made in the user's 
.bashrc file, which is typically located in the /home/user/ directory. 
The .bashrc file is a hidden file noted by its name starting with ".". 
Other

Here is an example .bashrc which demonstrates how to add permanent paths 
and how to define new functions within the operating system. 
For example, to copy the output of the pwd command to the system's 
clipboard, use cpyc then "ctrl-v" elsewhere. 

This file modifies environment variables such as 
\href{https://help.ubuntu.com/community/EnvironmentVariables}{PATH}  
when a bash terminal session is launched. Changes to environment 
variables are made when a new terminal session is started. The 
location of this file is typically: /home/user/.bashrc.  The 

\href{https://gist.github.com/kparrish/6064111}{.bashrc} from 
Kevin Parrish.


Shell scripts are relatively low-level sets of system commands 
which can manipulate environment variables (Section )   
and the Linux operating system. 

Here is a simple tutorial on writing shell scripts 
\href{http://linuxcommand.org/wss0010.php}
{http://linuxcommand.org/wss0010.php}

Running Linux system commands in Python can be an effective way of 
generating and manipulating many files with one script. Python 
is a more robust language than lower-level shell scripting. 
Here is an example:

lmp.in.iseed

%--------------------------------------------------------------------------
\subsection{\label{A:Comp_Env:Remote}Working on Remote Resources}
%--------------------------------------------------------------------------

At some point during research you will need to excecute code on remote 
resources which are typically large ($> 100$ cpu) computing clusters. 
You will be provided with a terminal session similar to the session 
provided by Ubuntu with most of the same system commands. 

While I recommend 
\href{https://filezilla-project.org/}{Filezilla} 
for handling the transfer of data and files, 
the functionality of Filezilla is contained in several shell 
commands:

ssh user@gilgamesh.cheme.cmu.edu

or equivalently by the machine's ip address:

ssh user@xxx.xxx.xxx.xxx

Files can be transferred using the following command:

scp -r user@gilgamesh.cheme.cmu.edu:/home/user/directory/ ./

which will place the "directory" and its contents into the pwd (./) 
of the local terminal session. 

There are many variants of the operating systems used for remote 
computing clusters, but the differences are usually superficial. 
During my work I used the 
\href{http://gilgamesh.cheme.cmu.edu/doc/gilgamesh.html}
{gilgamesh} computing cluster maintained by 
\href{https://github.com/jkitchin}{John Kitchin}. 
\href{http://gilgamesh.cheme.cmu.edu/doc/gilgamesh.html}
{Gilgamesh's documentation} is a good resource for learning how to run 
calculations on a computing cluster.

%--------------------------------------------------------------------------
\section{\label{A:Comp_Env:Programs}
Installing, Writing and Executing Programs}
%--------------------------------------------------------------------------

%--------------------------------------------------------------------------
\subsection{\label{A:Comp_Env:Install}Installing Available Packages}
%--------------------------------------------------------------------------

Before writing any of your own code, it is best to utilize any useful 
code which may already exist. There are numerous available packages 
for performing calculations in fields such as genetics, computational, 
fluid mechanics, and (of course) thermal transport modeling (see the 
following Section ). 


Most linux OS have automatic software installation. For Ubuntu, program  
management is achieved using the 
\href{https://help.ubuntu.com/community/AptGet/Howto}{apt-get} 
system command:

jason@jason-900X3C ~/ (master) $\$$ sudo apt-get install gfortran
[sudo] password for jason: 
Reading package lists... Done
Building dependency tree       
Reading state information... Done
gfortran is already the newest version.
The following package was automatically installed and is no longer required:
  kde-l10n-engb
Use 'apt-get autoremove' to remove it.
0 upgraded, 0 newly installed, 0 to remove and 129 not upgraded.

To check that the package has been installed, use:

jason@jason-900X3C-900X3D-900X3E-900X4C-900X4D ~/ (master) $\$$ which gfortran
/usr/bin/gfortran

This shows that gfortran program is installed in the /usr/bin/ folder, 
a common location where automatically installed programs are located. 
Because the 
folder /usr/bin is in the PATH environment variable:


the program gfortran is always available no matter where you are in 
the system terminal. 

To get more information about the gfortran, use:

jason@jason-900X3C ~/ (master) $\$$ ls -l /usr/bin/gfortran
lrwxrwxrwx 1 root root 12 Apr 22 03:44 /usr/bin/gfortran -> gfortran-4.7

where we see that /usr/bin/gfrotran "points" to gfortran-4.7. 
This is called a symbolic link, which can be created using:

jason@jason-900X3C ~/disorder/pcbm/topotools-tutorial-part2 (master) $\$$ ls -l /usr/bin/gfortran
lrwxrwxrwx 1 root root 12 Apr 22 03:44 /usr/bin/gfortran -> gfortran-4.7

%--------------------------------------------------------------------------
\subsection{\label{A:Comp_Env:Avail}Available Programs and Packages for 
Thermal Transport Modeling}
%--------------------------------------------------------------------------

Available progams represent oppurtunities to perform research quickly and 
easily. I suggest you read their documentation carefully and try the 
tutorials which are typically computationally inexpensive. 

Most of the packages listed below cannot be installed using Ubuntu's 
apt-get command, although 
\href{http://lammps.sandia.gov/download.html#ubuntu}
{this feature does exist for LAMMPS} and more packages are likely 
to be added in the future. It is important to read the installation 
documentation carefully. It is also helpful to contact any system 
linux/unix system administration staff or other researchers who are 
experienced with installing packages. These people can usually be 
found at campus or local computing centers, or in your own 
research group! 

For a first-time user, I reccomend trying to install the LAMMPS 
package as it is one of the easiest and most versatile. It has 
standard installation files for systems ranging from your 
personal computer up to massively-parallel supercomputer. 

Install lammps:

Set PATH for lammps

gnu compilers vs intel: gnu freely available
tar -zxvf lammps.tar, make serial, sudo apt-get install openmpi, make openmpi

open-source: 

\href{http://lammps.sandia.gov/}{LAMMPS}, including the particularly 
useful 
\href{http://lammps.sandia.gov/threads/threads.html}{mailing list} 
(use ctrl-f to search on any topic)  
and
\href{http://lammps.sandia.gov/doc/Section_python.html}
{python interface} which is used with the package 
\href{http://github.com/ntpl/ntpy}{ntpy}. 
\href{http://projects.ivec.org/gulp/}{GULP}, 
\href{http://www.abinit.org/}{ABINIT}, 
\href{http://www.quantum-espresso.org/}{Quantum Espresso}, 
\href{http://jp-minerals.org/vesta/en/}{VESTA}, 
\href{http://phonopy.sourceforge.net/}{phonopy}, 
\href{http://www.ks.uiuc.edu/Research/vmd/}{VMD} 
(with 
\href{https://sites.google.com/site/akohlmey/software/topotools}
{topotools})
\href{http://github.com/ntpl/ntpy}{ntpy},
\href{https://www.vasp.at/}{VASP}
\href{http://www.stfc.ac.uk/CSE/randd/ccg/software/DL_POLY/25526.aspx}
{DL$\_$POLY}, 
\href{http://icmab.cat/leem/siesta/}
{siesta}, 
\href{http://www.msg.ameslab.gov/gamess/}{GAMESS}, 
\href{http://www.cp2k.org/}{CP2K}, 
\href{http://www.icams.de/content/departments/ams/madsen/boltztrap.html}{BOLTZTRAP}, 
\href{http://www.homepages.ucl.ac.uk/~ucfbdxa/phon/}{PHON}, 

I reccomend trying this 
\href{https://gist.github.com/kparrish/6064159}{install.sh} created by 
\href{http://www.github.com/kparrish}{Kevin Parrish} which installs 
many programs and packages, including lammps for parallel computing 
with openmpi.

%--------------------------------------------------------------------------
\subsection{\label{A:Comp_Env:Writing}Writing Programs}
%--------------------------------------------------------------------------

its in this section \ref{A:coding_lang}

%--------------------------------------------------------------------------
\subsubsection{\label{A:coding_lang}
Coding Languages: Compiled versus Interpreted}
%--------------------------------------------------------------------------

There are many languages 
used for the open-source and lisenced packages(REF) that can be used 
to study nanoscale transport. These packages use 
\href{http://en.wikipedia.org/wiki/Compiled_language}{compiled} and 
\href{http://en.wikipedia.org/wiki/Interpreted_language}{interpreted}  
languages and often a miture of the two.

The most commonly used compiled languages are 
\href{https://en.wikipedia.org/wiki/C\%2B\%2B}{C/C++} 
(linux, LAMMPS) 
and 
\href{http://en.wikipedia.org/wiki/Fortran}{Fortran}
(GULP, quantum espresso, VASP).

A good discussion on 
\href{http://stackoverflow.com/questions/13078736/fortran-vs-c-does-fortran-still-hold-any-advantage-in-numerical-analysis-thes}
{the strengths of C++ versus Fortran}.

\href{http://www.youtube.com/watch?v=XFQ9dw3CyDo&list=PL1D10C030FDCE7CE0}
{excellent c++ tutorial}

\href{http://www.youtube.com/watch?v=YRTEOFMUTzw&list=PL6A8E21D2E86A0155}
{excellent fortran tutorial}


The two interpreted languages you are likely to use are 
\href{http://en.wikipedia.org/wiki/MATLAB}{matlab}  
and 
\href{http://en.wikipedia.org/wiki/Python_(programming_language)}{python}.

The key to maximizing the potential of interpreted languages is by 
using the built-in ``vector'' functions and operations provided by the 
\href{http://www.mathworks.com/help/matlab/matlab_prog/vectorization.html}
{matlab}  
and 
\href{http://faculty.washington.edu/rjl/uwamath583s11/sphinx/notes/html/python_vect.html}
{python}  
programming libraries.

matlab has an excellent built-in guide, google search will typically 
yield useful results. A good open-source substitute for matlab is 
\href{http://www.gnu.org/software/octave/}
{octave} which is capable to running most matlab scripts.

%--------------------------------------------------------------------------
\subsubsection{\label{A:coding_lang:case1}Case-study(a): Compiled 
Language, Lennard-Jones Argon Molecular Dynamics}
%--------------------------------------------------------------------------

The first case study is a 
\href{https://github.com/jasonlarkin/classes/tree/master/cmu/molecular_simulation/HW5}
{single C code to perform Molecular Dynamics 
on LJ argon}. The code uses simple subroutines and operates using 
a single (serial) processor. The 
\href{http://www.cplusplus.com/doc/tutorial/arrays/}
{arrays in this code are built statically}. 
Arrays can be created 
\href{http://www.cplusplus.com/doc/tutorial/dynamic/}{dynamically}, 
which allows for the input of systems with varying number of 
atoms. Addtionally, 
\href{http://www.cplusplus.com/reference/vector/vector/}{vectors} 
can be created and destroyed dynamically and have some advantages 
over arrays.  

The code is compiled using the 
\href{http://www.gnu.org/}{GNU project's} 
C++ compiler, \href{http://www.cprogramming.com/g++.html}{g++}:
\begin{lstlisting}
jason@jason-900X3C ~/disorder/md_serial (master) $ 
g++ ArgonMD.cpp -o ArgonMD_O_g++
\end{lstlisting}
The code can be run and the output can be directed using a useful 
shell operation called 
\href{http://www.linfo.org/pipes.html}{piping}, 
which is demonstrated below:
\begin{lstlisting}
jason@jason-900X3C ~/disorder/md_serial (master) $ 
ArgonMD_O_g++ > ./ArgonMD_O/output.txt
\end{lstlisting}
The code can be compiled using 
\href{http://gcc.gnu.org/onlinedocs/gcc/Optimize-Options.html#Optimize-Options}
{optimization flags, such as -O and -O3}:
\begin{lstlisting}
jason@jason-900X3C ~/disorder/md_serial (master) $ 
g++ -O3 ArgonMD.cpp -o ArgonMD_O3_g++
\end{lstlisting}
which greatly decreases the run time of this particular code. 
The total run time 
can dispalyed by using the shell command 
\href{http://linux.about.com/od/commands/l/blcmdl1_grep.htm}{grep},  
which shows for no optimization:
\begin{lstlisting}
jason@jason-900X3C ~/disorder/md_serial (master) $ 
grep -A 1 "Total Time" ./ArgonMD/output.txt
Total Time: 38.42 (s)
\end{lstlisting}
for -O optimization:
\begin{lstlisting}
jason@jason-900X3C ~/disorder/md_serial (master) $ 
grep -A 1 "Total Time" ./ArgonMD_O/output.txt
Total Time: 18.06 (s)
\end{lstlisting}
and for -O3 optimization:
\begin{lstlisting}
jason@jason-900X3C ~/disorder/md_serial (master) $ 
grep -A 1 "Total Time" ./ArgonMD_O3/output.txt
Total Time: 11.74 (s)
\end{lstlisting}
A useful shell command is 
\href{http://www.tuxfiles.org/linuxhelp/vimcheat.html}{vi}, a 
shell-based text editor, which can display the output within 
a shell terminal:
\begin{lstlisting}
jason@jason-900X3C ~/disorder/md_serial (master) 
$ vi ./ArgonMD/output.txt
0       1       5.61358 2.01864 5.26596
...
\end{lstlisting}

To compare with the above results, the lammps code is compiled in serial 
using no optimization, -O, and -O3.  
The results are contained in folders beginning with ``lmp''.  A useful 
shell function is 
\href{http://www.tuxfiles.org/linuxhelp/tabtrick.html}{tab-twice}, 
where tapping the tab key twice 
will display files and folders which begin with the same 
characters:
\begin{lstlisting}
jason@jason-900X3C ~/disorder/md_serial (master) $ 
grep -A 1 "Loop time" ./lmp
lmp.in.lj~     lmp_serial/    lmp_serial_O/  lmp_serial_O3/
\end{lstlisting}
The tab key can also be used for  
\href{http://en.wikipedia.org/wiki/Command-line_completion}
{command-line completion} to complete directory and file names. The run 
time for no optimization is:
\begin{lstlisting}
jason@jason-900X3C ~/disorder/md_serial (master) $ 
grep -A 1 "Loop time" ./lmp_serial/log.lammps
Loop time of 0.718188 on 1 procs for 1000 steps with 256 atoms

--
Loop time of 3.4892 on 1 procs for 5000 steps with 256 atoms
\end{lstlisting}
for -O optimization:
\begin{lstlisting}
jason@jason-900X3C ~/disorder/md_serial (master) $ 
grep -A 1 "Loop time" ./lmp_serial_O/log.lammps
Loop time of 0.19842 on 1 procs for 1000 steps with 256 atoms

--
Loop time of 0.917175 on 1 procs for 5000 steps with 256 atoms
\end{lstlisting}
and for -O3 optimization:
\begin{lstlisting}
jason@jason-900X3C ~/disorder/md_serial (master) $ 
grep -A 1 "Loop time" ./lmp_serial_O3/log.lammps
Loop time of 0.164066 on 1 procs for 1000 steps with 256 atoms

--
Loop time of 0.786311 on 1 procs for 5000 steps with 256 atoms
\end{lstlisting}

The decrease in run time with increasing optimization 
for lammps is roughly the same as for my code. However, for every 
optimization the lammps code is approximately an order of 
magnitude faster.  This is 
due to a several factors, the most important being 
the implementation of 
\href{http://en.wikipedia.org/wiki/Neighbor_list}{neighbor lists} 
as discussed in the 
\href{http://lammps.sandia.gov/doc/neighbor.html}
{lammps documentation}. The interested reader is encouraged to 
investigate the lammps code further for useful C++ coding 
practices. 

%--------------------------------------------------------------------------
\subsubsection{\label{A:coding_lang:case2}Compiled versus Interpreted 
Case-study(b): Lennard-Jones Dispersion}
%--------------------------------------------------------------------------

With interpreted languages traditional programming practice of using 
loops (for/do/while, etc) will slow the code down.  

Matlab version using mixture of loops and vectorized functions.

Fortran version (GULP) 

%--------------------------------------------------------------------------
\subsubsection{\label{A:coding_lang:case3}Compiled versus Interpreted 
Case-study(c): Allen-Feldman Diffusivity Calculation}
%--------------------------------------------------------------------------

Two systems, my local laptop:

http://www.samsung.com/us/computer/series-9-notebooks


\begin{lstlisting}
jason@jason-900X3C ~/disorder (master) $\$$ cat /proc/cpuinfo 
processor	: 0
vendor_id	: GenuineIntel
cpu family	: 6
model		: 58
model name	: Intel(R) Core(TM) i5-3317U CPU @ 1.70GHz
stepping	: 9
microcode	: 0x17
cpu MHz		: 782.000
cache size	: 3072 KB
physical id	: 0
siblings	: 4
core id		: 0
cpu cores	: 2
apicid		: 0
initial apicid	: 0
\end{lstlisting}


\href{http://gilgamesh.cheme.cmu.edu/doc/gilgamesh.html}{gilgamesh}



\begin{lstlisting}
jason@gilgamesh > cat /proc/cpuinfo
processor	: 0
vendor_id	: AuthenticAMD
cpu family	: 16
model		: 9
model name	: AMD Opteron(tm) Processor 6128 HE
stepping	: 1
cpu MHz		: 2000.003
cache size	: 512 KB
physical id	: 1
siblings	: 8
core id		: 0
cpu cores	: 8
apicid		: 16
\end{lstlisting}

%--------------------------------------------------------------------------
\subsection{\label{A:coding_lang:Execute}Executing Programs}
%--------------------------------------------------------------------------

%--------------------------------------------------------------------------
\subsubsection{\label{A:coding_lang:Execute:Local}
Executing Locally: Rapid Development}
%--------------------------------------------------------------------------

%--------------------------------------------------------------------------
\subsubsection{\label{A:coding_lang:Execute:Remote}
Executing Remotely: Portable Batch Systems}
%--------------------------------------------------------------------------

%--------------------------------------------------------------------------
\subsubsection{\label{A:coding_lang:Execute:Scaling}
Scaling Calculations}
%--------------------------------------------------------------------------

The majority of the methods 
used in this work scale poorly with the number of atoms, $N_a^\alpha$ 
with $\alpha >1$. 

Let's take the scaling cost $N_a^3$, which is the scaling for 
eigenvalue solution used in Sections . 
The cost of performing this calculation for a large system size 
$N_{a,large}$  
and every succesive system which is half the size of the former 
is given by the 
\href{http://en.wikipedia.org/wiki/Geometric_series}{geometric series} 
with common ratio $r=1/8$
\begin{equation}\label{EQ:cost_total}
\begin{split}
cost_{total} = N_{a,large}\frac{1}{1-r} = 1.143N_{a,large} 
\end{split}
\end{equation}
It costs a minimal amount ($ \approx 14\%$) to study systems smaller 
than the largest system considered. Even a linear scaling 
$N_a$ has $cost_{total} = 2N_{a,large}$. 
Because of this I recommend 
picking the system of maximum size and then start calcualtions with 
the smallest system of interest. 
Publication drafts can be 
developed mush faster by performing computationally cheap calculations 
first, documenting the results, and then iterating to more 
computationally expensive calculations. 
This scheme for performing calculations can follow these time-scales 
for calculation costs:  
one second, minute, hour, day, and week. I have performed 
countless calculations costing around one second, and very few which cost 
more than one week. Publication quality results will typically 
cost between one hour and one week.

%--------------------------------------------------------------------------


%%--------------------------------------------------------------------------
%\subsection{Importance of Open-Source Code}
%%--------------------------------------------------------------------------

%LAMMPS Mailing List
%http://lammps.sandia.gov/threads/threads.html

%search: 

%lennard (jones)	45
%silicon		125
%nanotube	153
%ger


%Execellent discussion of ensembles and newtonian dynamics
%http://lammps.sandia.gov/threads/msg13979.htmlmanium	2
%silica		79
%carbon		181
%hydrogen	99
%fullerene(C60)	7(12)	
%Pb (lead)	2
%water		541
%copper(cu)	25
%gold		81
%diamond		43

%phase transition	21
%green kubo	18


%Execellent discussion of ensembles and newtonian dynamics
%http://lammps.sandia.gov/threads/msg13979.html

%video of python development tree

%\href{http://www.youtube.com/watch?v=3poNeQHUKrs}
%{google+ history with gource}

%\href{http://www.youtube.com/watch?v=cNBtDstOTmA}
%{python history with gource}

%grand-daddy of them all, 
%\href{http://www.youtube.com/watch?v=pOSqctHH9vY}
%{linux kernel history with gource}

%It is important that results become more open-source.  It's important 
%that our communication becomes open-source. It's important that the 
%entire numerical process be carried out open-source. 


%%--------------------------------------------------------------------------
%\subsection{Redundancy}
%%--------------------------------------------------------------------------
%
%There are 3 different codes (shiomi/esfarjani, broido, ankit) for 
%performing the same basic calculations.
%
%For LD, there is GULP, phonopy, SIESTA, etc.
%
%%--------------------------------------------------------------------------
%\subsection{Code Development}
%%--------------------------------------------------------------------------
%
%took ankit 10 months to re-develop esfarjani code.  
%
%code development time can be drastically reduced using pre-existing code. 
%codes written in a modular fashion can be added to easily.
%
%%--------------------------------------------------------------------------
%\subsection{Experiment Pre-dating Simulation}
%%--------------------------------------------------------------------------
%The ideal goal of simulation is to pre-date experiment.  This has not 
%been achieved yet.  See Fig.
%
%findings for:
%CNT/Graphene
%Si
%Thermoeletric (LUC, alloys, SL, etc)
%Perovskites
%PCM
%
%%--------------------------------------------------------------------------
%\subsection{Experiment Pre-dating Simulation}
%%--------------------------------------------------------------------------
%
%ntpl most cited paper 1-5:
%maradudin, ladd, dove, ziman, 

%--------------------------------------------------------------------------
\section{Preparing Journal Articles and Thesis}
%--------------------------------------------------------------------------

reccomendation: student advisor should try and exchange editing a written 
research documen at least every week. The exchange of such a document 

Such a research document could be the running 
collection of journal articles which turn into the thesis. Maintenance 
of this document can be achieved with 
\href{https://www.dropbox.com/}{Dropbox} or 
Github. 
Github offers to advantage of smart version control and a built-in 
wiki. 



%--------------------------------------------------------------------------
\subsection{Journal Articles}
%--------------------------------------------------------------------------

The job of the student is to prepare, submit, and publish 
peer-reviewed journal articles. There are many journals suitable for 
nanoscale transport topics. All of them accept 
\href{http://www.latex-project.org/}{Latex} prepared manuscripts. 
I recommend the Latex editor 
\href{http://kile.sourceforge.net/}{Kile}, while the simple 
\href{https://projects.gnome.org/gedit/}{gedit} works well and comes 
pre-installed with Ubuntu. Here is a 
\href{http://mally.stanford.edu/~sr/computing/latex-example.html}
{simple latex example} and how to generate a 
\href{http://tex.stackexchange.com/questions/1596/how-to-compile-a-latex-document}
{portable document format (PDF)} from the latex document.  

To maintain the reference library I 
recommend 
\href{http://www.zotero.org/}{zotero}.  Here is an example 
\href{}{reference.bib} 
file which is exported automatically by zotero. 
The references are generated from the latex document using 
\href{http://www.bibtex.org/Using/}{bibtex} which compiles the contents of 
the reference.bib.

Here is an example of the latex files 
used to create an article published from this work:

\href{https://github.com/jasonlarkin/thesis/tree/master/thesis}
{Predicting alloy properties}.

This file uses 
\href{http://publish.aps.org/revtex}{revtex}, 
which is an article class used by Physical Review, 
Journal of Applied Physics, and others.

%--------------------------------------------------------------------------
\subsection{Thesis}
%--------------------------------------------------------------------------

These can be used as templates

Here is a Carnegie Mellon thesis template:

\href{https://github.com/robsimmons/cmu-thesis}
{https://github.com/robsimmons/cmu-thesis}

\href{https://github.com/jasonlarkin/thesis/tree/master/thesis}
{J. Larkin thesis files}.

\href{http://web.science.mq.edu.au/~rdale/resources/writingnotes/latexstruct.html}
{Article on structuring large documents.}

%2) Computational cost of each paper
%CPU, memory
%parallel cpu, parallel memory
%Trends: 
%small single cpu jobs -> large parallel cpu jobs
%single predictive method -> several predictive methods
%reccomendation: explore all predictive methods and models possible

%--------------------------------------------------------------------------
\subsection{Producing Figures}
%--------------------------------------------------------------------------

Python has a plotting module, \href{http://matplotlib.org/}{matplotlib}, 
which has \href{http://matplotlib.org/examples/index.html}{many examples}. 
Here is a \href{https://gist.github.com/kparrish/6101681}{simple example 
demonstrating how to generate and save data, load that data, and plot.} 


\href{https://github.com/ntpl/ntpy/tree/master/examples/thesis}
{Here is a simple example demonstrating how to }

%--------------------------------------------------------------------------
\section{Technical Advice}
%--------------------------------------------------------------------------

In addition to your advisor and close mentors, 
I reccomend communicating with experts in the field as much as 
possible without being an annoying.  How often to 
communicate depends on the situation. 
It is best to let the expert dictate the pace of the conversation. 

%--------------------------------------------------------------------------
\subsection{Expert Advice}
%--------------------------------------------------------------------------

Here is a list of experts I used as resources for this work. They 
will typically answer emails within 24-48 hours:


\href{http://ntpl.me.cmu.edu/people.html}
{Alan McGaughey},
\href{http://www.cmu.edu/me/malen/Lab_Website/People.html}
{Jon Malen},
\href{http://chemistry.curtin.edu.au/people/academic.cfm/J.Gale}
{Julian Gale} 
\href{http://projects.ivec.org/gulp/news.html}
{(GULP author)},
\href{http://mech.rutgers.edu/content/keivan-esfarjani}
{Keivan Esfarjani},
normand.mousseau@umontreal.ca,
guido.raos@polimi.it,
\href{mailto:joseph.feldman.ctr@nrl.navy.mil}
{Joseph Feldman},
\href{http://www.phonon.t.u-tokyo.ac.jp/}
{Junchiro Shiomi},
\href{http://www2.mpip-mainz.mpg.de/~donadio/tnt/People.html}
{Davide Donadio},
\href{http://www.ce.cmu.edu/people/faculty/maloney.html}
{Craig Maloney},
\href{http://quasiamore.mit.edu/pmwiki.php?n=Main.JivteshGarg}
{Jivtesh Garg},
\href{John Duda}
{John Duda},
\href{Wissam Al-Saidi}
{Wissam Al-Saidi},
\href{Dan Sellan}
{Dan Sellan},
\href{Ankit Jain}
{Ankit Jain},
\href{wong@andrew.cmu.edu}
{Wee-Liat Ong},
\href{John Kitchin}
{John Kitchin},
Steve Plimpton via the 
\href{http://lammps.sandia.gov/threads/threads.html}{LAMMPS mailing list} 
Axel Kolhmeyer via the 
\href{http://lammps.sandia.gov/threads/threads.html}{LAMMPS mailing list} 
and \href{akohlmey@gmail.com}{email}

\href{http://atztogo.users.sourceforge.net/}{Atz Togo}, creator of 
\href{http://phonopy.sourceforge.net/}{phonopy}

\href{http://jasonlarkin.github.io}{me}

%--------------------------------------------------------------------------
\subsection{Online Resources}
%--------------------------------------------------------------------------

\href{https://en.wikipedia.org/wiki/Lennard-Jones_potential}
{wikipedia}

\href{http://www.sklogwiki.org/}
{sklogwiki}

we can do better, needs to be organization. 

\href{http://nanohub.org/}{http://nanohub.org/}

does not provide good HPC resources. 




