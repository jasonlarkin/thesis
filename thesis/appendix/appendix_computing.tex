%--------------------------------------------------------------------------
\chapter{Setting Up Computing Environment}
%--------------------------------------------------------------------------

%--------------------------------------------------------------------------
\section{Setting Up Computing Environment}
%--------------------------------------------------------------------------

%--------------------------------------------------------------------------
\subsection{Setting Up Computing Environment}
%--------------------------------------------------------------------------

%--------------------------------------------------------------------------
\subsection{Suggested Reading}
%--------------------------------------------------------------------------


The online encyclopedia \href{http://www.wikipedia.org/}{wikipedia} 
has become an excellent resource for technical knowledge. 

\href{http://en.wikipedia.org/wiki/Statistical_mechanics}
{Statistical Mechanics}/
\href{http://en.wikipedia.org/wiki/Condensed_matter}
{Condensed Matter}
\cite{ashcroft_solid_1976,mcquarrie_statistical_2000}

\href{http://en.wikipedia.org/wiki/Phonon}
{Lattice Dynamics} with focus on the classical, quantum 
and thermodynamic properties of phonons.\cite{peierls_,ziman,
wallace_thermodynamics_1976,
srivastava_1990,dove_lattice_1993} 

\href{http://en.wikipedia.org/wiki/Introduction_to_quantum_mechanics}{Introduction to quantum mechanics and quantum chemistry}
\cite{griffiths_introduction_1995} 
and more 
advanced topics on 
\href{https://en.wikipedia.org/wiki/Electron_configuration}
{Electron Strucutre} and 
\href{http://en.wikipedia.org/wiki/Density_functional_theory}
{Density Functional Theory}.
\cite{martin_electronic_2004} 

Analytical methods for 
\href{http://en.wikipedia.org/wiki/Ordinary_differential_equations}
{ordinary} and 
\href{http://en.wikipedia.org/wiki/Partial_differential_equation}
{partial differential equations}, 
\href{http://en.wikipedia.org/wiki/Fourier_analysis}
{fourier} and 
\href{http://en.wikipedia.org/wiki/Statistics}
{statistical analysis},\cite{mcquarrie_mathematical_2003} 
and 
\href{http://en.wikipedia.org/wiki/Numerical_analysis}
{numerical analysis}.\cite{moin_fundamentals_2010} For example, 
the LJ MD code discussed in Section uses a  
\href{http://en.wikipedia.org/wiki/Verlet_integration}
{verlet integration method}. 

%--------------------------------------------------------------------------
\subsection{Setting Up Computing Environment}
%--------------------------------------------------------------------------

\href{http://www.youtube.com/watch?v=7XTHdcmjenI}
{Linux: world's most widely used software}

Linux runs on cell phones to the world's largest supercomputers. 

Reccomend the most widely used linux version 
\href{http://www.ubuntu.com/}{ubuntu}.
well-documented, large-community discussion

Apple OS is an adequate substitute as it is 
\href{https://en.wikipedia.org/wiki/Unix-like}{unix-like}.



Ubuntu is certified on 
\href{http://www.ubuntu.com/certification/}
{many top PC's from many computer companies}
There is even a company, \href{https://www.system76.com/}{system 76}, 
who builds ubuntu-based computers solely. I used the 
\href{https://www.system76.com/laptops/model/panp9}
{Pangolin Performance} for over 1 year of my PhD.  However, 
I reccomend using the lightest, most-portable, and longest-battery 
life notebook available such as the 
\href{https://help.ubuntu.com/community/SamsungSeries9}
{Samsung Series 9}  
or \href{https://help.ubuntu.com/community/MacBookAir}{Macbook Air}. 
You will typically be using this notebook to access large computing 
clusters so there is a benefit of having portability and long 
battery life over large computational power.    

%--------------------------------------------------------------------------
\subsection{Managing the Operating System}
%--------------------------------------------------------------------------

These instructions work well for linux and unix type OS

%--------------------------------------------------------------------------
\subsection{Installing Programs}
%--------------------------------------------------------------------------

Ubuntu and similar linux OS have automatic software installation and 
management using the apt-get command:

jason@jason-900X3C ~/ (master) $\$$ sudo apt-get install gfortran
[sudo] password for jason: 
Reading package lists... Done
Building dependency tree       
Reading state information... Done
gfortran is already the newest version.
The following package was automatically installed and is no longer required:
  kde-l10n-engb
Use 'apt-get autoremove' to remove it.
0 upgraded, 0 newly installed, 0 to remove and 129 not upgraded.


To check that the package has been installed, use:

jason@jason-900X3C-900X3D-900X3E-900X4C-900X4D ~/ (master) $\$$ which gfortran
/usr/bin/gfortran

This shows that gfortran program is installed in the /usr/bin/ folder, a common location where automatically installed programs are located. Because the 
folder /usr/bin is in the PATH environment variable:


the program gfortran is always available no matter where 

To get more information about the gfortran, use:

jason@jason-900X3C ~/ (master) $\$$ ls -l /usr/bin/gfortran
lrwxrwxrwx 1 root root 12 Apr 22 03:44 /usr/bin/gfortran -> gfortran-4.7

where we see that /usr/bin/gfrotran "points" to gfortran-4.7. This is called a symbolic link, which can be created using:

jason@jason-900X3C ~/disorder/pcbm/topotools-tutorial-part2 (master) $\$$ ls -l /usr/bin/gfortran
lrwxrwxrwx 1 root root 12 Apr 22 03:44 /usr/bin/gfortran -> gfortran-4.7

%--------------------------------------------------------------------------
\subsection{Available Programs}
%--------------------------------------------------------------------------

Low-hanging fruit.  Read their documentation, try the tutorials. 

Install lammps:

Set PATH for lammps

gnu compilers vs intel: gnu freely available
tar -zxvf lammps.tar, make serial, sudo apt-get install openmpi, make openmpi

open-source: 

\href{http://lammps.sandia.gov/}{LAMMPS}, including the particularly 
useful 
\href{http://lammps.sandia.gov/threads/threads.html}{mailing list} 
(use ctrl-f to search on any topic)  
and
\href{http://lammps.sandia.gov/doc/Section_python.html}
{python interface} which is used with the package 
\href{http://github.com/ntpl/ntpy}{ntpy}. 
\href{http://projects.ivec.org/gulp/}{GULP}, 
\href{http://www.abinit.org/}{ABINIT}, 
\href{http://www.quantum-espresso.org/}{Quantum Espresso}, 
\href{http://jp-minerals.org/vesta/en/}{VESTA}, 
\href{http://phonopy.sourceforge.net/}{phonopy}, 
\href{http://www.ks.uiuc.edu/Research/vmd/}{VMD} 
(with 
\href{https://sites.google.com/site/akohlmey/software/topotools}
{topotools})
\href{http://github.com/ntpl/ntpy}{ntpy},
\href{https://www.vasp.at/}{VASP}
\href{http://www.stfc.ac.uk/CSE/randd/ccg/software/DL_POLY/25526.aspx}
{DL_POLY}, 
\href{http://icmab.cat/leem/siesta/}
{siesta}, 
\href{http://www.msg.ameslab.gov/gamess/}{GAMESS}, 
\href{http://www.cp2k.org/}{CP2K}, 
\href{http://www.icams.de/content/departments/ams/madsen/boltztrap.html}{BOLTZTRAP}, 
\href{http://www.homepages.ucl.ac.uk/~ucfbdxa/phon/}{PHON}, 

I reccomend trying this 
\href{}{install.sh} created by 
\href{http://www.github.com/kparrish}{Kevin Parrish} which installs 
many programs and packages, including lammps for parallel computing 
with openmpi.

%--------------------------------------------------------------------------
\subsection{Environment Variables}
%--------------------------------------------------------------------------

\href{http://en.wikipedia.org/wiki/Environment_variable}
{environment variable} such as $\$$PATH.  Here is how we determine what 
PATH is set to:


Set PATH for lammps (see Section ).


Permanent changes to environment variables can be made in the user's 
.bashrc file, which is typically located in the /home/user/ directory. 
The .bashrc file is a hidden file noted by its name starting with ".". 
Other

Here is an example .bashrc which demonstrates how to add permanent paths 
and how to define new functions within the operating system. 
For example, to copy the output of the pwd command to the system's 
clipboard, use cpyc then "ctrl-v" elsewhere. 

The path to programs can also be added to an existing path in the PATH 
variable by using symbolic links.  Here is an example:

%--------------------------------------------------------------------------
\subsection{.bashrc}
%--------------------------------------------------------------------------

This file modifies environment variables such as 
\href{}{PATH}  
when a bash terminal session is launched. Changes to environment 
variables are made when a new terminal session is started. The 
location of this file is typically: /home/user/.bashrc.  The 
``.'' 

%--------------------------------------------------------------------------
\subsection{System Commands}
%--------------------------------------------------------------------------

Useful Linux Commands:

sed, grep, ls, cd, pwd, export, setenv, scp -r, ssh, sudo, nohup, 
vi, cat, which, echo

\href{https://www.google.com/search?q=linux+sed}
{google: linux sed}


%--------------------------------------------------------------------------
\subsection{Shell Scripts}
%--------------------------------------------------------------------------

Shell scripts are relatively low-level scripts which execute system 
commands (Section ) and 
manipulate environment variables (Section )   
of the linux/unix operating system. 

Here is a simple tutorial on writing shell scripts 
\href{http://linuxcommand.org/wss0010.php}
{http://linuxcommand.org/wss0010.php}

%--------------------------------------------------------------------------
\subsection{Remote Resources Commands}
%--------------------------------------------------------------------------

I reccomend Filezilla for handling the transfer of data and files.  
However, the functionality of Filezilla is contained in several shell 
commands:

ssh user@gilgamesh.cheme.cmu.edu

or equivalently by the machine's ip address:

ssh user@xxx.xxx.xxx.xxx

Files can be transferred using the following command:

scp -r user@gilgamesh.cheme.cmu.edu:/home/user/directory/ ./

which will place the "directory" and its contents into the pwd (./) 
of the local shell. 

Here are useful PBS cluster commands:

qstat, qdel, 
 

%--------------------------------------------------------------------------
\subsection{System Commands in Python}
%--------------------------------------------------------------------------

Running Linux System Commands in Python
 
python vs other languages: combines strengths of the 3 important:
PERL+java+matlab
1) shell-scripting: controlling the OS, system(cmd), (PERL) 

lmp.in.iseed

2) hi-level organization: object-oriented (module.method(input)) (Java)
3) lin alg + numerical methods: numpy.method, scipy.method (fortran, matlab)
python deficiencies
more about this in the appendix

%--------------------------------------------------------------------------
\subsection{Preparing Journal Articles and Thesis}
%--------------------------------------------------------------------------

http://tex.stackexchange.com/questions/1596/how-to-compile-a-latex-document

http://stackoverflow.com/questions/2461905/compiling-latex-bib-source

http://publish.aps.org/revtex

used 

http://kile.sourceforge.net/

http://www.zotero.org/

matplotlib to generate figures. 

%--------------------------------------------------------------------------
\subsection{Preparing Journal Articles}
%--------------------------------------------------------------------------

The job of the student is to prepare, submit, and (hopefully) publish 
peer-reviewed journal articles. There are many journals suitable for 
nanoscale transport topics.  Here is one example of the latex files 
used to create an article used in this work:

This file uses Revtex, which is an article class used by Physical Review, 
Journal of Applied Physics, and others. The paper can be compiled using:



These can be used as templates



Here is a Carnegie Mellon thesis template:

https://github.com/robsimmons/cmu-thesis

Useful article on structuring large documents

http://web.science.mq.edu.au/~rdale/resources/writingnotes/latexstruct.html

https://github.com/jasonlarkin/thesis/tree/master/thesis


1) Use latex
- reccomend Kile editor, gedit works well too. 
- reccomend zotero to maintain library.bib

1) Time to prepare each paper

reccomendation: student advisor should try and exchange editing a written 
research document every day, at least every week.  

Such a research document could be the running 
collection of journal articles which turn into the thesis. Maintenance 
of this document can be achieved with Dropbox or Github.  Github 
offers to advantage of smart version control and internet wiki. 

2) Computational cost of each paper

CPU, memory
parallel cpu, parallel memory

Trends: 
small single cpu jobs -> large parallel cpu jobs
single predictive method -> several predictive methods

reccomendation: explore all predictive methods and models possible

%--------------------------------------------------------------------------
\subsection{Advice}
%--------------------------------------------------------------------------

In addition to your advisor and close mentors, 
I reccomend communicating with experts in the field as much as 
possible without being an annoyance.  How often to 
communicate depends on the situation. 
It is best to let the expert dictate the pace of the conversation. 

%--------------------------------------------------------------------------
\subsection{Expert Advice}
%--------------------------------------------------------------------------

Here is a list of experts I used as resources for this work. They 
will typically answer emails within 24-48 hours:


\href{http://ntpl.me.cmu.edu/people.html}
{Alan McGaughey},
\href{http://www.cmu.edu/me/malen/Lab_Website/People.html}
{Jon Malen},
\href{http://chemistry.curtin.edu.au/people/academic.cfm/J.Gale}
{Julian Gale} 
\href{http://projects.ivec.org/gulp/news.html}
{(GULP author)},
\href{http://mech.rutgers.edu/content/keivan-esfarjani}
{Keivan Esfarjani},
normand.mousseau@umontreal.ca,
guido.raos@polimi.it,
\href{mailto:joseph.feldman.ctr@nrl.navy.mil}
{Joseph Feldman},
\href{http://www.phonon.t.u-tokyo.ac.jp/}
{Junchiro Shiomi},
\href{http://www2.mpip-mainz.mpg.de/~donadio/tnt/People.html}
{Davide Donadio},
\href{http://www.ce.cmu.edu/people/faculty/maloney.html}
{Craig Maloney},
\href{http://quasiamore.mit.edu/pmwiki.php?n=Main.JivteshGarg}
{Jivtesh Garg},
\href{John Duda}
{John Duda},
\href{Wissam Al-Saidi}
{Wissam Al-Saidi},
\href{Dan Sellan}
{Dan Sellan},
\href{Ankit Jain}
{Ankit Jain},
\href{wong@andrew.cmu.edu}
{Wee-Liat Ong},
\href{John Kitchin}
{John Kitchin},
Steve Plimpton via the 
\href{http://lammps.sandia.gov/threads/threads.html}{LAMMPS mailing list} 
Axel Kolhmeyer via the 
\href{http://lammps.sandia.gov/threads/threads.html}{LAMMPS mailing list} 
and \href{akohlmey@gmail.com}{email}

\href{http://atztogo.users.sourceforge.net/}{Atz Togo}, creator of 
\href{http://phonopy.sourceforge.net/}{phonopy}

\href{http://jasonlarkin.github.io}{me, haha}


%--------------------------------------------------------------------------
\subsection{Coding Languages}
%--------------------------------------------------------------------------

As we will see in the dicussion following, there are many languages 
used for the open-source and lisenced packages(REF) that can be used 
to study nanoscale transport.
%--------------------------------------------------------------------------
\subsection{Compiled versus Interpreted}
%--------------------------------------------------------------------------

http://en.wikipedia.org/wiki/Compiled_language

http://en.wikipedia.org/wiki/Interpreted_language

The most commonly used compiled languages are 
\href{https://en.wikipedia.org/wiki/C\%2B\%2B}{C++} 
(linux, LAMMPS, spglib) 
and 
\href{http://en.wikipedia.org/wiki/Fortran}{Fortran}: 
(GULP, quantum espresso, VASP)

A good discussion on 
\href{http://stackoverflow.com/questions/13078736/fortran-vs-c-does-fortran-still-hold-any-advantage-in-numerical-analysis-thes}
{the strengths of C++ versus Fortran}.

\href{http://www.youtube.com/watch?v=XFQ9dw3CyDo&list=PL1D10C030FDCE7CE0}
{excellent c++ tutorial}

\href{http://www.youtube.com/watch?v=YRTEOFMUTzw&list=PL6A8E21D2E86A0155}
{excellent fortran tutorial}


The two most likely interpreted languages you are likely to use are 
matlab (licesensed) http://en.wikipedia.org/wiki/MATLAB 
and 
python (open-source) http://en.wikipedia.org/wiki/Python_(programming_language).



The key to maximizing the potential of these languages is by vectorization: 
using the built-in ``vector'' functions and operations provided by the 
matlab http://www.mathworks.com/help/matlab/matlab_prog/vectorization.html 
and 
python http://faculty.washington.edu/rjl/uwamath583s11/sphinx/notes/html/python_vect.html 
programming libraries.

matlab has an excellent built-in guide, google search will typically 
yield useful results. 

\href{http://www.gnu.org/software/octave/}
{matlab}

\href{http://www.gnu.org/software/octave/}
{octave}



%--------------------------------------------------------------------------
\subsection{Case-study 1: LJ Argon Molecular Dynamics}
%--------------------------------------------------------------------------

The first case study is a 
\href{https://github.com/jasonlarkin/classes/tree/master/cmu/molecular_simulation/HW5}
{single C code to perform Molecular Dynamics 
on LJ argon}. The code is a serial code with 

/home/jason/classes/cmu/molecular_simulation/HW5

\lstset{language=Shell}
\begin{lstlisting}[label=some-code,caption=Some Code]
jason@jason-900X3C ~/classes/cmu/molecular_simulation/HW5 (master) c++ ArgonMD.cpp -o ArgonMD
\end{lstlisting}

\lstset{language=Shell}
\begin{lstlisting}[label=some-code,caption=Some Code]
jason@jason-900X3C ~/classes/cmu/molecular_simulation/HW5 (master) ./ArgonMD
\end{lstlisting}

The output is.  A useful shell operation is 
\href{http://www.linfo.org/pipes.html}{"piping"}, which is demonstrated below:

\lstset{language=Shell}
\begin{lstlisting}[label=some-code,caption=Some Code]
jason@jason-900X3C ~/classes/cmu/molecular_simulation/HW5 (master) $ ./ArgonMD > props_piped.txt
\end{lstlisting}

Another useful shell command is 
\href{http://www.tuxfiles.org/linuxhelp/vimcheat.html}{"vi"}, a 
shell-based text editor:
\lstset{language=Shell}
\begin{lstlisting}[label=some-code,caption=Some Code]
jason@jason-900X3C ~/classes/cmu/molecular_simulation/HW5 (master) $ vi props_piped.txt
5.61358 2.01864 5.26596
3.4073  1.72339 1.32187
5.28007 0.107792        4.95407
1.45285 5.09976 0.961697
4.5679  4.82261 5.78293
5.29423 0.880753        0.709709
3.63248 5.45978 2.70637
...
\end{lstlisting}

%--------------------------------------------------------------------------
\subsection{Case-study 1: LJ Dispersion}
%--------------------------------------------------------------------------

With interpreted languages traditional programming practice of using 
loops (for/do/while, etc) will slow the code down.  

Matlab version using mixture of loops and vectorized functions.

Fortran version (GULP) 

%--------------------------------------------------------------------------
\subsection{Case-study 1: Allen-Feldman Diffusivity Calculation}
%--------------------------------------------------------------------------

Two systems, my local laptop:

http://www.samsung.com/us/computer/series-9-notebooks

\lstset{language=Shell}
\begin{lstlisting}[label=some-code,caption=Some Code]
jason@jason-900X3C ~/disorder (master) $\$$ cat /proc/cpuinfo 
processor	: 0
vendor_id	: GenuineIntel
cpu family	: 6
model		: 58
model name	: Intel(R) Core(TM) i5-3317U CPU @ 1.70GHz
stepping	: 9
microcode	: 0x17
cpu MHz		: 782.000
cache size	: 3072 KB
physical id	: 0
siblings	: 4
core id		: 0
cpu cores	: 2
apicid		: 0
initial apicid	: 0
\end{lstlisting}


and gilgamesh

http://gilgamesh.cheme.cmu.edu/doc/gilgamesh.html

\lstset{language=Shell}
\begin{lstlisting}[label=some-code,caption=Some Code]
jason@gilgamesh > cat /proc/cpuinfo
processor	: 0
vendor_id	: AuthenticAMD
cpu family	: 16
model		: 9
model name	: AMD Opteron(tm) Processor 6128 HE
stepping	: 1
cpu MHz		: 2000.003
cache size	: 512 KB
physical id	: 1
siblings	: 8
core id		: 0
cpu cores	: 8
apicid		: 16
\end{lstlisting}

%--------------------------------------------------------------------------
\subsection{Case-study 1: Normal Mode Decomposition}
%--------------------------------------------------------------------------

Shows uses shared memory arrays and multi-processing on these arrays. 




cython: example LAMMPS python wrapper
f2py: simple example. suggest warpper be built for GULP.

%--------------------------------------------------------------------------
\subsection{Scaling Calculations}
%--------------------------------------------------------------------------

combination of increasing hardware, improving software.  The 
improvement of these aspects depends partly on current 
research.  

Have most control over what calculations

Time-scales for data points: 
1 sec, 1 min, 1 hour, 1 day, 1 week, 1 month, 1 year 
number of calculations per PhD:
1E8 , 1E6 , 1E4, 1E3 , 1E2 , 1E1 , 1

1) Run system sizes by doubling. Most calcualtions scale poorly (i.e., 
(SYS_SIZE^alpha with alpha > 1). Pick a maximum system size based 
on computatinal resources and time constraints.  Run all system sizes 
halved before running maximum system size. 

2) Run all trivial (< 5 mins) calculations on a given system using all 
methods possible. 

Explore all trivial calculations first. Increase the time-scale and 
decrease the numner of data-point accordingly. 

reccomendation: 
%--------------------------------------------------------------------------


%--------------------------------------------------------------------------
\subsection{Online Resources}
%--------------------------------------------------------------------------

wikipedia

\href{http://www.sklogwiki.org/SklogWiki/index.php/Main_Page}
{octave}

https://en.wikipedia.org/wiki/Lennard-Jones_potential

http://www.sklogwiki.org/SklogWiki/index.php/Lennard-Jones_model

we can do better, needs to be organization. 

http://nanohub.org/

does not provide good HPC resources. 


%--------------------------------------------------------------------------
\subsection{Importance of Open-Source Code}
%--------------------------------------------------------------------------

LAMMPS Mailing List
http://lammps.sandia.gov/threads/threads.html

search: 

lennard (jones)	45
silicon		125
nanotube	153
ger


Execellent discussion of ensembles and newtonian dynamics
http://lammps.sandia.gov/threads/msg13979.htmlmanium	2
silica		79
carbon		181
hydrogen	99
fullerene(C60)	7(12)	
Pb (lead)	2
water		541
copper(cu)	25
gold		81
diamond		43

phase transition	21
green kubo	18


Execellent discussion of ensembles and newtonian dynamics
http://lammps.sandia.gov/threads/msg13979.html

video of python development tree

\href{http://www.youtube.com/watch?v=3poNeQHUKrs}
{google+ history with gource}

\href{http://www.youtube.com/watch?v=cNBtDstOTmA}
{python history with gource}

grand-daddy of them all, 
\href{http://www.youtube.com/watch?v=pOSqctHH9vY}
{linux kernel history with gource}

%It is important that results become more open-source.  It's important 
%that our communication becomes open-source. It's important that the 
%entire numerical process be carried out open-source. 


%%--------------------------------------------------------------------------
%\subsection{Redundancy}
%%--------------------------------------------------------------------------
%
%There are 3 different codes (shiomi/esfarjani, broido, ankit) for 
%performing the same basic calculations.
%
%For LD, there is GULP, phonopy, SIESTA, etc.
%
%%--------------------------------------------------------------------------
%\subsection{Code Development}
%%--------------------------------------------------------------------------
%
%took ankit 10 months to re-develop esfarjani code.  
%
%code development time can be drastically reduced using pre-existing code. 
%codes written in a modular fashion can be added to easily.
%
%%--------------------------------------------------------------------------
%\subsection{Experiment Pre-dating Simulation}
%%--------------------------------------------------------------------------
%The ideal goal of simulation is to pre-date experiment.  This has not 
%been achieved yet.  See Fig.
%
%findings for:
%CNT/Graphene
%Si
%Thermoeletric (LUC, alloys, SL, etc)
%Perovskites
%PCM
%
%%--------------------------------------------------------------------------
%\subsection{Experiment Pre-dating Simulation}
%%--------------------------------------------------------------------------
%
%ntpl most cited paper 1-5:
%maradudin, ladd, dove, ziman, 

